%
% Hanna Döring
% Benedikt Rednoss
% Matthias Schulte
% Xiaochuan Yang
% Dhandapani Yogeshwaran 
%
% Journal of Complex Networks
% Chaos. An Interdisciplinary Journal of Nonlinear Science
% Communications in Nonlinear Science and Numerical Simulation
% Markov Processes and Related Fields
% Journal of Physics. A. Mathematical and Theoretical
% Mathematical Methods in the Applied Sciences
% ALEA. Latin American Journal of Probability and Mathematical Statistics
% IEEE Control Systems Letters
% Theory of Probability and its Applications
% IEEE Communications Letters
%
% Discrete Mathematics
% Reviews of Modern Physics
% SIAM Journal on Discrete Mathematics
% Physical Review E
% Probability in the Engineering and Informational Sciences
% Entropy. An International and Interdisciplinary Journal of Entropy and Information Studies
% INFOR. Information Systems and Operational Research
% Journal of Statistical Mechanics: Theory and Experiment
% Theory of Probability and Mathematical Statistics
%
% Electronic Journal of Combinatorics
%
% Clarify when exactly the connectedness of G is needed
% 
% JAP
% RSA
% Journal of Complexity 
% Scientific reports
% Bernoulli
%
\documentclass[12pt]{article}
\usepackage{amsfonts}
%\usepackage{newtxmath}

% \usepackage[utf8]{inputenc}

% \usepackage[LGR,T1]{fontenc}
% \usepackage{selinput} % Auto-detect input encoding.
\usepackage{textalpha} % Enable Greek text
% \usepackage[greek,english]{babel}

% \SelectInputEncodingList{utf8, iso-8859-7}
% \SelectInputMappings{alpha=α}
% \SelectInputMappings{lambda=λ}

\usepackage{bm}
\usepackage{enumerate} 
\usepackage{amssymb,amsmath}
\usepackage{caption}
\usepackage{subcaption}
\usepackage{accents} 

\usepackage{dsfont}

\usepackage{stackengine}

\usepackage{accents} 

\let\Horig\H

\usepackage{tikz}
\usetikzlibrary{automata,topaths}
\usetikzlibrary{shapes}
\usetikzlibrary{plotmarks}

\usepackage{fvextra}

% \usepackage{wasysym} 
\usepackage{float} 
\usepackage{textcomp} 
\usepackage{siunitx}
\sisetup{output-exponent-marker=\ensuremath{\mathrm{e}}}

\usepackage{color} 
\definecolor{lightblue}{rgb}{0,0.2,0.5}
\usepackage[colorlinks=true, urlcolor=lightblue,linkcolor=lightblue, citecolor=lightblue]{hyperref}
\usepackage{ucs}
%% For typesetting code listings                                                

% \usepackage{listings}
\usepackage{listingsutf8}
\lstdefinelanguage{Sage}[]{Python}
{morekeywords={False,sage,True},sensitive=true}

\lstset{
 mathescape = false,
 escapechar = {$},
 basicstyle = \ttfamily,
 extendedchars=false,
 inputencoding=utf8,
%  frame=none,
%  showtabs=False,
%  showspaces=False,
%  showstringspaces=False,
  commentstyle={\ttfamily\color{dgreencolor}},
  keywordstyle={\ttfamily\color{dbluecolor}\bfseries},
  stringstyle={\ttfamily\color{dgraycolor}\bfseries},
  language=Sage,
  basicstyle={\fontsize{9pt}{9pt}\ttfamily},
  aboveskip=0.3em,
  belowskip=0.1em,
  numbers=none, % left,
  numberstyle=\footnotesize,
  breaklines=true,                % sets automatic line breaking 
% postbreak=\mbox{\textcolor{red}{$\hookrightarrow$}\space},
  }
  
\definecolor{dblackcolor}{rgb}{0.0,0.0,0.0}
\definecolor{dbluecolor}{rgb}{0.01,0.02,0.7}
\definecolor{dgreencolor}{rgb}{0.2,0.4,0.0}
\definecolor{dgraycolor}{rgb}{0.30,0.3,0.30}
\newcommand{\dblue}{\color{dbluecolor}\bf}
\newcommand{\dred}{\color{dredcolor}\bf}
\newcommand{\dblack}{\color{dblackcolor}\bf}

\usepackage{fvextra}

\usepackage{graphicx}
\usepackage{flushend,cuted}
\usepackage{bm}
\usepackage{tabularx}
%\usepackage{color}
\usepackage{indentfirst}
\usepackage{amssymb}
\usepackage{xparse}
\usepackage{tikz}
\usepackage{mdwlist}
\usepackage{tkz-graph}

\DeclareMathAlphabet{\eufrak}{U}{}{}{} 
\SetMathAlphabet\eufrak{normal}{U}{euf}{m}{n}
\SetMathAlphabet\eufrak{bold}{U}{euf}{b}{n}


% \usepackage{eucal}

% \usepackage{epsfig,latexsym}

% \usepackage{graphicx,amsmath,amssymb,latexsym,psfrag}

% \usepackage[polish]{babel}

\newtheorem{assumption}{Assumption}[section]

% \usepackage{graphics,graphicx,amsmath}

\oddsidemargin=0cm \textwidth=16.5cm \textheight=23cm
\topmargin=-1.5cm
\newcommand{\R}{\mathbb{R}}
\newcommand{\V}{\mathbb{V}}
\newcommand{\T}{\mathbb{T}}
\newcommand{\C}{\mathbb{C}}
\newcommand{\E}{\mathbb{E}}
\newcommand{\IP}{\mathbb{P}}
%\newcommand{\bone}{\bone}
\newcommand{\bone}{{\bf 1}}
% \newcommand{\E}{\mathrm{E}}

\newcommand{\supp}{\mathrm{supp}}

%% \renewcommand{\le}{\leqslant}
%% \renewcommand{\leq}{\leqslant}

%% \renewcommand{\ge}{\geqslant}
%% \renewcommand{\geq}{\geqslant}

\newcommand{\conv}{\mathrm{conv}}
\newcommand{\card}{\mathrm{card}}
\newcommand{\grad}{\mathrm{grad}}
\newcommand{\N}{\mathbb{N}}
\newcommand{\inte}{\mathbb{N}}
\newcommand{\bbf}{{\mathbf{f}}}
\newcommand{\bT}{\mathbb{T}}
\newcommand{\TP}{\widetilde{P}}
\newcommand{\tgi}{t\rightarrow \infty}
\newcommand{\ngi}{n\rightarrow \infty}
\newcommand{\algi}{\alpha \rightarrow \infty}
\newcommand{\xgi}{x\rightarrow \infty}
\newcommand{\oJ}{\overline{J}}
\newcommand{\og}{\overline{\gamma}}
\newcommand{\oL}{\overline{\Lambda}}
\newcommand{\EPSI}{\varepsilon}

% To define \widebar
\makeatletter
\newcommand*\rel@kern[1]{\kern#1\dimexpr\macc@kerna}
\newcommand*\widebar[1]{%
  \begingroup
  \def\mathaccent##1##2{%
    \rel@kern{0.8}%
    \overline{\rel@kern{-0.8}\macc@nucleus\rel@kern{0.2}}%
    \rel@kern{-0.2}%
  }%
  \macc@depth\@ne
  \let\math@bgroup\@empty \let\math@egroup\macc@set@skewchar
  \mathsurround\z@ \frozen@everymath{\mathgroup\macc@group\relax}%
  \macc@set@skewchar\relax
  \let\mathaccentV\macc@nested@a
  \macc@nested@a\relax111{#1}%
  \endgroup
}
\makeatother

\makeatletter
\DeclareRobustCommand\widecheck[1]{{\mathpalette\@widecheck{#1}}}
\def\@widecheck#1#2{%
    \setbox\z@\hbox{\m@th$#1#2$}%
    \setbox\tw@\hbox{\m@th$#1%
       \widehat{%
          \vrule\@width\z@\@height\ht\z@
          \vrule\@height\z@\@width\wd\z@}$}%
    \dp\tw@-\ht\z@
    \@tempdima\ht\z@ \advance\@tempdima2\ht\tw@ \divide\@tempdima\thr@@
    \setbox\tw@\hbox{%
       \raise\@tempdima\hbox{\scalebox{1}[-1]{\lower\@tempdima\box
\tw@}}}%
    {\ooalign{\box\tw@ \cr \box\z@}}}
\makeatother

\newcommand{\disc}{\mathrm{disc}}
\newcommand{\bZ}{\bold{Z}}
\newcommand{\bz}{\bold{z}}
\newcommand{\dtv}{{d_{\rm TV}}}
\newcommand{\dk}{{d_{\rm K}}}
\newcommand{\dw}{{d_{\rm W}}}


\newtheorem{prop}{Proposition}[section]
\newtheorem{lemma}[prop]{Lemma}
\newtheorem{definition}[prop]{Definition}
\newtheorem{corollary}[prop]{Corollary}
\newtheorem{thm}[prop]{Theorem}
\newtheorem{remark}[prop]{Remark}
\newtheorem{example}[prop]{Example}

%\def\bone{\vmathbb{1}}
\def\bp{\noindent{\it Proof.}\ }
\def\ep{\hfill $\Box$}
\newcommand{\bt}{\mathbf{t}}
\def\({\left(}
\def\){\right)}
% \theoremstyle{definition}
% \newtheorem{definition}{Definicja}[section]
\newcommand{\cov}{\mathrm{Cov}}
\def\[{\left[}
\def\]{\right]}
\def\real{{\mathord{\mathbb R}}}
\def\N{{\mathord{\mathbb N}}}
\def\Dom{\mathrm{Dom}}
\def\Var{\mathrm{Var}}
% \newcommand{\p}{\mathbb{P}}
\newcommand{\pr}{\mathbb{P}}
\def\P{\mathbb{P}}
% \newcommand{\R}{\mathbb{R}}
%\newcommand{\N}{\mathbb{N}}
\newcommand{\Z}{\mathbb{Z}}
% \newcommand{\C}{\mathbb{C}}
% \newcommand{\E}{\mathbb{E}}

\newcommand{\EE}{\mathsf{E}}
\newcommand{\re}{\mathbb{e}}

\newenvironment{Proof}{\removelastskip\par\medskip
\noindent{\em Proof.} \rm}{\penalty-20\null\hfill$\square$\par\medbreak}

\newenvironment{Proofx}{\removelastskip\par\medskip
\noindent{\em Proof.} \rm}{\par}

\newenvironment{Proofy}{\removelastskip\par\medskip
\noindent{\em Proof} \rm}{\penalty-20\null\hfill$\square$\par\medbreak}

\allowdisplaybreaks

\numberwithin{equation}{section}

% \usepackage{refcheck}
%%%%%%%%%%%%%%%%%%%%%%%%%%%%%% for drawing pictures by tikz
\GraphInit[vstyle = Shade]
\usetikzlibrary[intersections,
positioning,
petri,
backgrounds,
fit,
decorations.pathmorphing,
arrows,
arrows.meta,
bending,
calc,
intersections,
through,
backgrounds,
shapes.geometric,
quotes,
matrix,
trees,
shapes.symbols,
graphs,
math,
patterns,
external,
scopes,
matrix,
lindenmayersystems,
shapes.callouts,
shapes.misc,
angles,
shapes.arrows,
shadings]
%%%%%%%%%%%%%%%%%%%%%%%%%%%%%%%%%%%%%%%%
\tikzset{snake it/.style={-stealth,
decoration={snake, 
    amplitude = .4mm,
    segment length = 2mm,
    post length=0.9mm},decorate}}

\usetikzlibrary{matrix,calc}

\newcommand{\convexpath}[2]{
[   
    create hullnodes/.code={
        \global\edef\namelist{#1}
        \foreach [count=\counter] \nodename in \namelist {
            \global\edef\numberofnodes{\counter}
            \node at (\nodename) [draw=none,name=hullnode\counter] {};
        }
        \node at (hullnode\numberofnodes) [name=hullnode0,draw=none] {};
        \pgfmathtruncatemacro\lastnumber{\numberofnodes+1}
        \node at (hullnode1) [name=hullnode\lastnumber,draw=none] {};
    },
    create hullnodes
]
($(hullnode1)!#2!-90:(hullnode0)$)
\foreach [
    evaluate=\currentnode as \previousnode using \currentnode-1,
    evaluate=\currentnode as \nextnode using \currentnode+1
    ] \currentnode in {1,...,\numberofnodes} {
-- ($(hullnode\currentnode)!#2!-90:(hullnode\previousnode)$)
  let \p1 = ($(hullnode\currentnode)!#2!-90:(hullnode\previousnode) - (hullnode\currentnode)$),
    \n1 = {atan2(\y1,\x1)},
    \p2 = ($(hullnode\currentnode)!#2!90:(hullnode\nextnode) - (hullnode\currentnode)$),
    \n2 = {atan2(\y2,\x2)},
    \n{delta} = {-Mod(\n1-\n2,360)}
  in 
    {arc [start angle=\n1, delta angle=\n{delta}, radius=#2]}
}
-- cycle
}

\tikzset{hide labels/.style={every label/.append style={text opacity=0}}}

\makeatletter
\lst@InputCatcodes
\def\lst@DefEC{%
 \lst@CCECUse \lst@ProcessLetter
  ^^80^^81^^82^^83^^84^^85^^86^^87^^88^^89^^8a^^8b^^8c^^8d^^8e^^8f%
  ^^90^^91^^92^^93^^94^^95^^96^^97^^98^^99^^9a^^9b^^9c^^9d^^9e^^9f%
  ^^a0^^a1^^a2^^a3^^a4^^a5^^a6^^a7^^a8^^a9^^aa^^ab^^ac^^ad^^ae^^af%
  ^^b0^^b1^^b2^^b3^^b4^^b5^^b6^^b7^^b8^^b9^^ba^^bb^^bc^^bd^^be^^bf%
  ^^c0^^c1^^c2^^c3^^c4^^c5^^c6^^c7^^c8^^c9^^ca^^cb^^cc^^cd^^ce^^cf%
  ^^d0^^d1^^d2^^d3^^d4^^d5^^d6^^d7^^d8^^d9^^da^^db^^dc^^dd^^de^^df%
  ^^e0^^e1^^e2^^e3^^e4^^e5^^e6^^e7^^e8^^e9^^ea^^eb^^ec^^ed^^ee^^ef%
  ^^f0^^f1^^f2^^f3^^f4^^f5^^f6^^f7^^f8^^f9^^fa^^fb^^fc^^fd^^fe^^ff%
  ^^^^03b6% <--- for ζ
  ^^^^03b1^^^^03b2^^^^03b3%
  ^^00}
\lst@RestoreCatcodes
\makeatother

\begin{document}
\title{
\huge
% Subgraph counting with fixed endpoints in the random-connection model
 Graph connectivity with fixed endpoints in the random-connection model
} 

\author{
  Qingwei Liu\footnote{\href{mailto:qingwei.liu@ntu.edu.sg}{qingwei.liu@ntu.edu.sg}}
  \qquad
      Nicolas Privault\footnote{
\href{mailto:nprivault@ntu.edu.sg}{nprivault@ntu.edu.sg}
}
  \\
\small
Division of Mathematical Sciences
\\
\small
School of Physical and Mathematical Sciences
\\
\small
Nanyang Technological University
\\
\small
21 Nanyang Link, Singapore 637371
}

\maketitle

\vspace{-0.5cm}

\begin{abstract} 
We consider the count of subgraphs with an arbitrary configuration of endpoints in the random-connection model based on a Poisson point process on $\real^d$. We present combinatorial expressions for the computation of the cumulants and moments of all orders of such subgraph counts, which allow us to estimate the growth of cumulants as the intensity of the underlying Poisson point process goes to infinity. As a consequence, we obtain a central limit theorem with explicit convergence rates under the Kolmogorov distance, and connectivity bounds. Numerical examples are presented using a computer code in SageMath for the closed-form computation of cumulants of any order, for any type of connected subgraph and for any configuration of endpoints in any dimension $d\geq 1$. In particular, graph connectivity estimates, Gram-Charlier expansions for density estimation, and correlation estimates for joint subgraph counting are obtained. 
\end{abstract}
\noindent\emph{Keywords}:~
Random-connection model, 
subgraph count,
normal approximation,
Kolmogorov distance,
cumulant method,
Poisson point process,
random graphs,
connectivity.

\noindent 
{\em Mathematics Subject Classification:} 
60D05, % Geometric probability and stochastic geometry
05C80, % Random graphs (graph-theoretic aspects)
60G55, % Point processes (e.g., Poisson, Cox, Hawkes processes)
60F05. % Central limit and other weak theorems
% 60G57	Random measures
%60B10. % Convergence of probability measures
 
\baselineskip0.7cm

\section{Introduction}
\noindent
This paper considers the statistics and asymptotic behavior of
subgraph counts in a multidimensional random-connection
model based on a Poisson point process, 
 which can be used to model physical systems in e.g.
 statistical mechanics 
% \cite{georgiou2013,georgiou2015,georgiou2016,kartungiles2016},
 \cite{kartungiles2016}, 
 wireless networks
 \cite{ta2007,Mao2010,georgiou2015}, % mao-ng2010,
% complex networks, \cite{mulder2018,boguna2020},
% \cite{bianconi2001,betz2008,biskup2015}, 
 or cosmology \cite{cunningham2017,fountoulakis2020}. 

 \medskip

 The random-connection model is a natural generalization of the % random geometric graph and
 Erd\H os-R\'enyi random graph,
 in which vertices are randomly located and connected
 with location-dependent probabilities. 
 More precisely, given $\mu$ a % finite
 diffuse Radon measure on $\R^d$, the random-connection model $G_H (\eta )$
 consists of an underlying Poisson point process $\eta$ on $\R^d$
 with intensity of the form $\lambda \mu(\mathrm{d}x)$, $\lambda >0$,  
 in which any two vertices $x,y$ in $\eta $ are connected
 with the probability $H(x,y)$, 
 where $H:\real^d \times \real^d \to [0,1]$ is a given connection function. 
% Recently, a Central Limit Theorem has been derived in \cite{can2022} for the counts of induced subgraphs in the random-connection model under certain stabilization and moment conditions.
 
\medskip 

The count of subgraphs that connect any single point 
$x$ in the Poisson process $\eta$
to $m$ fixed endpoints $y_1,\ldots , y_m\in \real^d$
 is known to have a Poisson distribution with mean
$% \displaystyle
 \lambda \int_{\real^d} H(x,y_1)\cdots H(x,y_m)\mu (dx)$,
 see \cite{giles-privault2_published}
 where this Poisson property has been used
 to derive closed-form estimates of two-hop
 connectivity in the random-connection model
 when $m=2$. 

\medskip 
 
 In this paper, we consider the count of general connected
 subgraphs with a general configuration of
 fixed endpoints at fixed locations $y_1 , \ldots,y_m \in \R^d$ 
 in the random-connection model  
 $G_H (\eta \cup \{y_1, \ldots ,y_m\})$ constructed on the union
 of the Poisson point process $\eta$ and $\{ y_1, \ldots ,y_m\}$.
 In particular, we extend the subgraph count cumulant formulas obtained on 
 $G_H (\eta )$ in \cite{LiuPrivault} into joint cumulant expressions on
 $G_H (\eta \cup \{y_1, \ldots ,y_m\})$. 
 
 \medskip 

In Proposition~\ref{mom-cumfor} we derive
moment and cumulant expressions
for the count $N^G_{y_1,\ldots , y_m}$ of subgraphs with fixed endpoints
$y_1, \ldots ,y_m$ in $G_H (\eta\cup \{y_1, \ldots ,y_m\})$.
Such expressions allow us to determine the dominant terms in the growth of
 cumulants as the intensity $\lambda$ of the underlying point process tends to infinity, 
 by estimating the counts of vertices and edges in connected
 partition diagrams as in e.g. \cite{khorunzhiy}. 
 As a consequence, in Theorem~\ref{khopone}
 we obtain growth estimates for the cumulants of
 the subgraph count $N^G_{y_1,\ldots , y_m}$. 

 \medskip

 This allows us to show the convergence of renormalized subgraph
 counts to the normal distribution in Proposition~\ref{fjlfa12}
 as the intensity $\lambda$ of the underlying Poisson point process on $\R^d$
 tends to infinity.
Convergence rates under the Kolmogorov distance
are then obtained in Proposition~\ref{pkol}
for the normal approximation of subgraph counts
from the combinatorics of cumulants
 and the {Statulevi\v{c}ius condition}, see \cite{rudzkis,doering}
 and Lemma~\ref{Statuleviciuscond1}, 
 extending the results obtained in \cite{LiuPrivault}
 for subgraphs without endpoints.
 See also \cite{thale18} % \cite{grotethale18,thale18} 
 for other applications of this condition
 to concentration inequalities,
 normal approximation and moderate deviations for random polytopes.

 \medskip
  
 Connectivity probability
 estimates and bounds are derived using the second moment
 method in Proposition~\ref{jklf3}  
 and the factorial moment expansions in Proposition~\ref{fdshkf0}. 
% , see \eqref{fjkl21}. 
 Using third order cumulant expressions, we also provide
 improved fits of probability density functions of renormalized
 subgraph counts when the Gaussian approximation is not valid,
 see Figure~\ref{fig5}. 
 
\medskip 

In Section~\ref{examples} we consider several examples of
subgraphs with endpoints such as $k$-hop paths,
triangles and trees, for which  
exact cumulant computations are matched to their
Monte Carlo estimates 
 using the Rayleigh connection function $H(x,y) = e^{ - \beta \Vert x - y\Vert^2}$, 
 $\beta > 0$.
 In those examples
 we obtain graph connectivity estimates,
 Gram-Charlier expansions for density estimation,
 and correlation estimates for joint graph counting,
 which are matched to the outputs of Monte Carlo simulations. 
 Computations are done in closed form using symbolic calculus 
 in the SageMath coding implementations 
 presented in Appendices~\ref{fjkldsf}-\ref{fjkldsf-2}, 
 and available for download at 
 \url{https://github.com/nprivaul/random-connection}.  
 We note that although intensive computations may be required,
the types of connected subgraphs and associated configurations 
of endpoints considered is only limited by the available computing
power. 

% Those formulas are applied to different examples including $k$-hop paths with one and two endpoints, and cycles with endpoints. We also apply our cumulant estimates to probability density estimation using Gram-Charlier estimates. 

% \medskip 
% In \cite{giles-privault2}, $2$-hop connectivity with two fixed endpoints has been considered. 

\medskip 

This paper is organised as follows.
 Section~\ref{diagramrepresentation} introduces some
preliminaries on graphs and partition diagrams.
In Section~\ref{rcm}, we compute the cumulants
 of the counts of subgraphs with endpoints in the random-connection
model using summations over partitions. 
Subgraph count asymptotics and the associated
central limit theorem are given 
in Section~\ref{sca}, and numerical examples are
presented in Section~\ref{examples}.
A general derivation of joint cumulant identities is given 
in Appendix~\ref{appendixa}, extending the construction of
 \cite{LiuPrivault} from the univariate to the multivariate case. 
% As an application, we state a Central Limit Theorem with Kolmogorov convergence rate for the count of $k$-hop paths with one and two endpoints.
% In Section~\ref{appl-engineer} we consider some engineering applications, and
 Basic results on Gram-Charlier expansions and probability approximation
 using cumulant and moment methods are recalled in 
Appendices~\ref{statuleviciuscond} and \ref{s5}. 
% The SageMath codes for the computation of cumulants and joint cumulants are listed in Appendices~\ref{fjkldsf} and \ref{fjkldsf-2}.

% and \begin{equation}\label{neighborhood1} E_i :=\left\{j\in [m] \ : \ \{ v_i , e_j \} \in E_G\right\}, \qquad i=1, \ldots ,r. \end{equation} 
%
%
%
%\section{Normal approximation of subgraph counts on $G_H (\eta_A)$}
% \subsubsection*{Poisson point process} 
\section{Partition diagrams} 
\label{diagramrepresentation}
\noindent 
% Let $G=(V_G,E_G)$ be a graph with vertex set $V_G$ and edge set $E_G$.
% For any $u,v\in V_G$, we say $u,v$ are adjacent in $G$ if $\{u,v\}\in E$, and in this case we write $u\sim v$.
% A subgraph of $G$ is a graph $G'=(V_{G'},E_{G'})$ such that $V_{G'}\subset V_G$ and $E'\subset E$, and $G'$ is an induced subgraph of $G$ if $E'$ consists of all edges of $G$ having both endpoints in $V_{G'}$.
Given $r\geq 2$ and $m\geq 1$, we 
consider a connected graph $G=(V_G,E_G)$ with edge set $E_G$ and
vertex sequence 
$V_G=(v_1, \ldots ,v_r; e_1,\ldots , e_m)$, such that
\begin{enumerate}[i)]
\item the subgraph induced by $G$ on $\{v_1, \ldots ,v_r\}$ is connected, and 
\item $e_1, \ldots ,e_m$ are not adjacent to each other in $G$. 
\end{enumerate}
\noindent
 An example of such graph is described in Figure~\ref{fig:diagram0}, 
 with $r=4$ and $m=2$. 

\begin{figure}[H]
  \centering
  \begin{tikzpicture}
\draw[black, thick] (-1,1) rectangle (6,3);
\filldraw [gray] (1,2) circle (2pt);
\node[font=\small] at (1,2.4) {$v_1$};
\filldraw [gray] (2,2) circle (2pt);
\node[font=\small] at (2,2.4) {$v_2$};
\filldraw [gray] (3,2) circle (2pt);
\node[font=\small] at (3,2.4) {$v_3$};
\filldraw [gray] (4,2) circle (2pt);
\node[font=\small] at (4,2.4) {$v_4$};
\filldraw [gray] (-0,2) circle (2pt);
\node[font=\small] at (-0.5,2) {$e_1$};
\filldraw [gray] (5,2) circle (2pt);
\node[font=\small] at (5.5,2) {$e_2$};
\draw[thick,blue] (1,2) .. controls (1.5,2.3) .. (2,2);
\draw[thick,blue] (2,2) .. controls (2.5,2.3) .. (3,2);
\draw[thick,blue] (2,2) .. controls (3,1.7) .. (4,2);
\draw[thick,blue] (1,2) -- (0,2);
\draw[thick,blue] (4,2) -- (5,2);
\end{tikzpicture}
\caption{
 Graph $G=(V_G,E_G)$ with $V_G=(v_1, v_2,v_3,v_4;e_1,e_2)$, 
 $n=3$, $r=4$, $m=2$.}
\label{fig:diagram0}
\end{figure}
\vspace{-0.3cm}
% so that $|B_i^{[r]}|$ and $|B_i^{[m]}|$ represent the in-degree and out-degree of the vertex $v_i$, respectively.
\noindent
For $n\geq 1$ we let $[n]:=\{1, \ldots ,n\}$, and 
for any set $A$ we denote by $\Pi (A)$
 the collection of all set partitions of $A$.
 We also let $|A|$ denote the number of elements of any finite set $A$,
 in particular $|\sigma|$ represents the number of blocks in a partition
 $\sigma\in\Pi ([n]\times[r])$.
 In Definition~\ref{fjklf},
 to any graph $G$ and set partition $\rho\in\Pi ([n]\times[r])$
we associate a graph $\rho_G$ whose vertices are the blocks of $\rho$.
% For this, we generalize the construction of \cite{LiuPrivault} from $G_H (\eta)$ to the random-connection graph $G_H (\eta \cup \{y_1, \ldots ,y_m\})$ constructed on the union of $\eta$ with the points $\{ y_1, \ldots ,y_m\}$.
% Recall that $G=(V,E_G)$ is a connected graph with vertex set $V=\{v_1, \ldots ,v_{r+m}\}$ satisfying that the subgraph induced by $V_1:=\{v_1, \ldots ,v_r\}$ remains connected and $v_{r+1}, \ldots ,v_{r+m}$ are not adjacent with each other.
% We denote by $G_U=(U,E_U)$ the subgraph of $G$ induced by $U=\{v_1, \ldots ,v_r\}$.
 % , and for ease of notation we set $\Gamma(\eta_{A}):=\Gamma(\eta_{y_1, \ldots ,y_m})$ when $A=\{y_1, \ldots ,y_m\}\subset \R^d$.
 % , and assume that the blocks of $\rho= ( a_1, \ldots ,a_\ell ) \in\Pi[n\times r]$ are listed according to the lexicographic order.
\begin{definition}
   \label{fjklf}
   Given $\rho$ a partition of $[n]\times[r]$
   and $G=(V_G,E_G)$ a connected graph 
   on $V_G=(v_1, \ldots ,v_r; e_1,\ldots , e_m)$, 
   we let $\rho_G$ denote the graph 
   % from $n\times r + m$ nodes denoted respectively by $(i,j) \in [n] \times [r]$, and $(j) \in [m]$,
   constructed as follows on $[m] \cup [n]\times [r]$:
\begin{enumerate}[i)]  
\item for all $j_1, j_2\in [r]$, $j_1\not= j_2$, and $i\in [n]$, 
  an edge links $(i,j_1)$ to $(i,j_2)$
  iff $\{v_{j_1},v_{j_2}\}\in E_G$. 
\item for all $(j,k)\in [r]\times [m]$ and $i\in [n]$, an edge
  links $(k)$ to $(i,j)$ iff $\{v_j,e_k\}\in E_G$; 
\item for all $i_1,i_2\in [n]$
  and $j_1,j_2\in [r]$,
  merge any two nodes $(i_1,j_1)$ and $(i_2,j_2)$ 
  if they belong to a same block in $\rho$;  
\item eliminating any redundant edges created by the above construction.
\end{enumerate}
\end{definition}
\noindent
 If $\rho\in\Pi ([n]\times[r])$
 takes the form $\rho = \{ b_1,\ldots , b_{|\rho |}\}$, 
 the graph $\rho_G$ forms a connected graph with
 $|\rho | + m$ vertices, and we reindex the set of vertices $V_{\rho_G}$
 of $\rho_G$  
 as $V_{\rho_G}=[|\rho | + m ]$ according to the lexicographic order
 on $\inte \times \inte$, 
 followed by the remaining $m$ single-digit vertices, 
 indexed as $\{|\rho |+1,\ldots , |\rho | +m\}$,
 see Figure~\ref{fig:diagram1}-$b)$
 in which we have $|\rho | =9$, $m = 2$, and $V_{\rho_G}=(1,\dots ,9;10, 11)$. 

 \medskip

 \noindent
{\bf Example}. 
Take $r=4$, $m=2$ and $V_G=(v_1, v_2,v_3,v_4;e_1,e_2)$. 
 Figure~\ref{fig:diagram1}-$b)$  
 shows the graph $\rho_G$ defined from 
 $G=(V_G,E_G)$ of Figure~\ref{fig:diagram0}
 and the $9$-block partition $\rho \in \Pi ([3]\times[4])$
 given by 
 \begin{align*}
   \rho = \big\{ & \{(1,1)\},
   \\
   & \{(1,2),(2,2)\},
      \\
   & \{(1,3)\},
   \\
   & \{(1,4)\},
   \\
   & \{(2,1),(3,1)\},
   \\
   & \{(2,3)\},
   \\
   & \{(2,4),(3,4)\},
   \\
   & \{(3,2)\},
   \\
   & \{(3,3)\}\big\}. 
\end{align*} 
 
 % \vspace{0.3cm}

\begin{figure}[H]
\captionsetup[subfigure]{font=footnotesize}
\centering
\subcaptionbox{Diagram before merging edges and vertices.}[.5\textwidth]{%
\begin{tikzpicture}
\draw[black, thick] (0,0) rectangle (7,4);
\foreach \i in {1,2,3}
{
\filldraw [gray] (2,\i) circle (2pt);
\filldraw [gray] (3,\i) circle (2pt);
\filldraw [gray] (4,\i) circle (2pt);
\filldraw [gray] (5,\i) circle (2pt);
\draw[thick, dash dot,blue] (2,\i) .. controls (2.5,\i) .. (3,\i);
\draw[thick, dash dot,blue] (4,\i) .. controls (3.5,\i) .. (3,\i);
\draw[thick, dash dot,blue] (1,2) .. controls (1.5,\i) .. (2,\i);
\draw[thick, dash dot,blue] (6,2) .. controls (5.5,\i) .. (5,\i);
}

\draw[thick, dash dot,blue] (3,1) .. controls (4,1+.4) .. (5,1);
\draw[thick, dash dot,blue] (3,2) .. controls (4,2-.4) .. (5,2);
\draw[thick, dash dot,blue] (3,3) .. controls (4,3-.4) .. (5,3);

\node[anchor=north,font=\tiny] at (2,1) {(3,1)};
\node[anchor=north,font=\tiny] at (3,1) {(3,2)};
\node[anchor=north,font=\tiny] at (4,1) {(3,3)};
\node[anchor=north,font=\tiny] at (5,1) {(3,4)};
\node[anchor=south,font=\tiny] at (2,3) {(1,1)};
\node[anchor=south,font=\tiny] at (3,3) {(1,2)};
\node[anchor=south,font=\tiny] at (4,3) {(1,3)};
\node[anchor=south,font=\tiny] at (5,3) {(1,4)};
\node[anchor=south,font=\tiny] at (2,2) {(2,1)};
\node[anchor=south,font=\tiny] at (3,2) {(2,2)};
\node[anchor=south,font=\tiny] at (4,2) {(2,3)};
\node[anchor=south,font=\tiny] at (5,2) {(2,4)};

\filldraw [gray] (1,2) circle (2pt);
\node[anchor=east,font=\tiny] at (1,2) {$(1)$};
\filldraw [gray] (6,2) circle (2pt);
\node[anchor=west,font=\tiny] at (6,2) {$(2)$};
\draw[thick] (3,3) -- (3,2);
\draw[thick] (2,2) -- (2,1);
\draw[thick] (5,2) -- (5,1);
\end{tikzpicture}}%
\subcaptionbox{Graph $\rho_G$ after merging edges and vertices.}[.5\textwidth]{
\begin{tikzpicture}
\draw[black, thick] (0,0) rectangle (7,4);
\foreach \i in {3}
{
\filldraw [gray] (2,\i) circle (2pt);
\filldraw [gray] (3,\i) circle (2pt);
\filldraw [gray] (4,\i) circle (2pt);
\filldraw [gray] (5,\i) circle (2pt);
\draw[thick,blue] (2,\i) .. controls (2.5,\i) .. (3,\i);
\draw[thick,blue] (4,\i) .. controls (3.5,\i) .. (3,\i);
\draw[thick,blue] (3,\i) .. controls (4,\i-.4) .. (5,\i);
\draw[thick,blue] (1,2) .. controls (1.5,\i) .. (2,\i);
\draw[thick,blue] (6,2) .. controls (5.5,\i) .. (5,\i);
}
\node[anchor=south,font=\tiny] at (2,3) {1};
\node[anchor=south,font=\tiny] at (3,3) {2};
\node[anchor=south,font=\tiny] at (4,3) {3};
\node[anchor=south,font=\tiny] at (5,3) {4};
\node[anchor=south,font=\tiny] at (2,2) {5};
\node[anchor=south,font=\tiny] at (4,2) {6};
\node[anchor=south,font=\tiny] at (5,2) {7};
\node[anchor=north,font=\tiny] at (3,1) {8};
\node[anchor=north,font=\tiny] at (4,1) {9};
\node[anchor=north,font=\tiny] at (1,2) {10};
\node[anchor=north,font=\tiny] at (6,2) {11};
\filldraw [gray] (1,2) circle (2pt);
\filldraw [gray] (6,2) circle (2pt);
\filldraw [gray] (2,2) circle (2pt);
\filldraw [gray] (4,2) circle (2pt);
\filldraw [gray] (5,2) circle (2pt);
\filldraw [gray] (3,1) circle (2pt);
\filldraw [gray] (4,1) circle (2pt);
\draw[thick,blue] (1,2) .. controls (1.5,2) .. (2,2);
\draw[thick,blue] (3,3) .. controls (2.5,2.5) .. (2,2);
\draw[thick,blue] (3,3) .. controls (3.5,2.5) .. (4,2);
\draw[thick,blue] (3,3) .. controls (4,2.5) .. (5,2);
\draw[thick,blue] (5,2) -- (6,2);
\draw[thick,blue] (2,2) -- (3,1) -- (4,1);
\draw[thick,blue] (3,1) -- (5,2);

\end{tikzpicture}}%
\caption{
  Example of graph $\rho_G$
  % on $G=(V_G,E)$ with $E_G=\{v_1, v_2,v_3,v_4,e_1,e_2\}$ 
 with $n=3$, $r=4$, and $m=2$.}
\label{fig:diagram1}
\end{figure}

\vspace{-0.4cm}

% \medskip 

\begin{definition}
  For $\rho\in\Pi ([n]\times[r])$ of the form
  $\rho = \{ b_1,\ldots , b_{|\rho |}\}$ 
  and $j \in [m]$, we let 
\begin{equation}
\nonumber
    {\cal A}^\rho_j:=\{ k \in [ |\rho | ] \ : \ \exists (s,i)\in b_k ~\mathrm{s.t.}~
    (v_i,e_j) \in E_G % , \ s\in [n], \ i\in [r]
    \} 
\end{equation} 
denote the neighborhood of the vertex $(|\rho | + j)$ in $\rho_G$,
$j=1,\ldots , m$.
% where $k$ is the index of the vertex represented by the block $b_k$, $j=1,\ldots , m$. 
 \end{definition}
 For example, in the graph $\rho_G$ of Figure~\ref{fig:diagram1}
 we have ${\cal A}^\rho_1=\{1,5\}$ and ${\cal A}^\rho_2=\{4,7\}$. 
 \begin{definition}
   \label{def-1}
   Given $n,r\geq 1$,
  let $\pi:=\{\pi_1, \ldots ,\pi_n\}$ be the partition in $\Pi ([n]\times[r])$
  given by 
  $$
  \pi_i:=\left\{(i,1), \ldots ,(i,r)\right\},
  \quad
  i=1, \ldots , n.
  $$
 \begin{enumerate}[i)]
   \item A set partition $\sigma\in\Pi ([n]\times[r])$ is connected if $\sigma\vee\pi=\widehat{1}$, 
     where
     $\sigma \vee\pi$ is the finest set partition which is coarser than both
     $\sigma$ and $\pi$, and $\widehat{1} = \{ [n]\times [r] \}$
is the coarsest partition of $[n]\times [r]$. 
\item 
 A set partition $\sigma\in\Pi ([n]\times[r])$ is non-flat if $\sigma\wedge\pi=\widehat{0}$,
 where
 $\sigma \wedge\pi$ is
 the coarsest set partition which is finer than both $\sigma$ and $\pi$,
 and $\widehat{0}$ is the finest partition of $[n]\times [r]$.
\end{enumerate} 
 We let $\Pi_{\widehat{1}} ([n]\times[r])$ denote the collection of all
connected partitions of $[n] \times [r]$. 
\end{definition}
\section{Cumulants of subgraph counts with endpoints} 
\label{rcm}
\noindent
In what follows we consider 
 a Radon measure $\mu$ on $\real^d$, and we let 
$\IP_\lambda$, $\lambda > 0$, 
 denote the distribution of the Poisson point process $\eta$ 
 with intensity $\lambda \mu (dx)$ on the space 
 $$\mathcal{C}:=\left\{\omega\subset\R^d \ : \ |\omega\cap A|<\infty ~\text{for any bounded set $A\subset\R^d$}\right\}  
 $$
 of locally finite configurations on $\R^d$. 
 In other words, 
 \begin{enumerate}[i)] 
 \item for any relatively compact Borel set $B\subset \R^d$, the distribution of $\eta(B)$ under $\IP_\lambda$ is Poisson with parameter $\lambda \mu (B)$.
   \vspace{-0.2cm}  
% $\mu mathrm{Vol}(B)$ stands for the volume of the set $B$, 
 \item for any $n\geq 2$ and pairwise disjoint
   relatively compact Borel sets $B_1, \ldots ,B_n\subset\R^d$, the random variables $\eta(B_1), \ldots ,\eta(B_n)$ are independent under $\IP_\lambda$.
\end{enumerate}
% \subsubsection*{Set partitions} \noindent
% Let $\Pi (b)$ be the collection of all set partitions of set $b$, and $|\sigma|$ the number of blocks in the partition $\sigma\in\Pi(b)$.
% Given two set partitions $\sigma_1,\sigma_2\in\Pi(b)$, we say that $\sigma_1$ is coarser than $\sigma_2$, or equivalently, $\sigma_2$ is finer than $\sigma_1$, written as $\sigma_2\preceq\sigma_1$ if any block in $\sigma_1$ is a combination of blocks in $\sigma_2$.
% We denote by $\sigma_1\vee\sigma_2$ the finest set partition which is coarser than both $\sigma_1$ and $\sigma_2$, and by $\sigma_1\wedge\sigma_2$ the coarsest set partition that is finer than $\sigma_1$ and $\sigma_2$. 
% We say that $u,v$ are connected, denote as $u\leftrightarrow v$, if there is a path between $u$ and $v$, i.e. there exist $v_1, \ldots ,v_n\in V$ such that $u\sim v_1\sim \cdots v_n \sim v$. 
% The graph $G$ is connected if for any two vertices $u,v\in V$ there is a path from $u$ to $v$.
% Two graphs $G=(V,E)$ and $G'=(V',E')$ are isomorphic if there is a bijection $T:V\to V'$ such that $\{u,v\}\in E$ if and only if $\{T(u),T(v)\}\in E'$ for any $u\ne v\in V$. We write $G\simeq G'$ to indicate $G,G'$ are isomorphic.
%\begin{definition}
\noindent
 Given $H:\R^d\times \R^d\to[0,1]$ 
 a symmetric connection function and  
 $y_1 , \ldots,y_m$ fixed points in $\R^d$,
 the random-connection model 
  $G_H (\eta \cup \{y_1, \ldots ,y_m\})$
 is the random graph built on the union of $\{y_1 , \ldots,y_m \}$
 and a Poisson point process
 sample $\eta$, 
  in which any two distinct points 
  $x,y\in \eta \cup \{y_1, \ldots ,y_m\}$
  are independently connected by an edge with the probability $H(x,y)$.  
% depending on the locations of $x$ and $y$. 
% \end{definition}
% Given $H:\R^d\times \R^d\to[0,1]$ a symmetric measurable connection function, we assume that any two distinct points $x,y\in\real^d$ are independently connected by an edge with the probability $H(x,y)$, in which case we write $x\leftrightarrow y$. 
% \subsubsection*{Graph notation} 
\begin{definition}
\label{fjkl} 
  Given $m$ fixed points $y_1 , \ldots,y_m \in \R^d$,
  for a.s. $\eta$ we let $N_{y_1,\ldots , y_m}^G$ denote the count of subgraphs
 in $G_H (\eta \cup \{y_1,\ldots , y_m \} )$
 that are isomorphic to $G=(V_G,E_G)$ in the sense that 
 there exists a (random) injection
 from $V_G$ into $\eta \cup \{y_1,\ldots , y_m \}$
 which is one-to-one from $\{e_1,\ldots , e_m\}$ to $\{y_1,\ldots , y_m\}$, 
 and preserves the graph structure of $G$. 
\end{definition}
\noindent
According to Definition~\ref{fjkl}, we express the subgraph count
 $N^G_{y_1,\ldots , y_m}$ as  
\begin{equation}
\nonumber
  N_{y_1,\ldots , y_m}^G=\sum_{(x_1, \ldots ,x_r)\in\eta^{r}}f_{y_1,\ldots , y_m} (x_1, \ldots ,x_r), 
\end{equation}
where the random
function $f:(\real^d)^r \to \{0,1\}$ defined as 
\begin{equation}
\nonumber
f_{y_1,\ldots , y_m} (x_1, \ldots ,x_r):=
\prod_{
  \substack{
    1 \leq i \leq r
    \\
    1 \leq j \leq m
    \\ \{v_i,e_j\}\in E_G }
}
\bone_{\{y_j\leftrightarrow x_i\}} 
\prod_{\substack{ 1 \leq k,l \leq r
    \\ \{v_\ell,v_k\}\in E_G}}\bone_{\{x_\ell\leftrightarrow x_k\}},
\qquad
 x_1,\ldots , x_r \in \R^d, 
\end{equation}
is independent of the Poisson point process $\eta$,
 and $\bone_{\{x\leftrightarrow y\}}=1$ if and only if
$x\neq y$ and $x,y\in \real^d$ are connected in the
 random-connection model 
 $G_H (\eta \cup \{y_1, \ldots ,y_m\})$.

\medskip 

 % Definition~\ref{fjklf} allows us to rewrite the moment and cumulant formulas
% of Propositions~\ref{fjl} and \ref{connectedcumulant-1}
 The following partition summation formulas 
 extend \cite[Proposition~5.1]{LiuPrivault}
 to the counting of subgraphs with endpoints,
 and they are a special case of 
 Proposition~\ref{fjklf2} in appendix,
 which deals with joint subgraph counting. 
\begin{prop}
\label{mom-cumfor}
 The moments and cumulants of $N_{y_1,\ldots , y_m}^G$ admit
 the following expressions: 
$$
  \E_\lambda \big[\big(N_{y_1,\ldots , y_m}^G\big)^n\big]=
  \sum_{\substack{\rho\in\Pi ([n]\times[r])
      \\\rho\wedge\pi=\widehat{0}} \atop {\rm (non-flat)}}
  \lambda^{|\rho |}
  \int_{(\R^d)^{|\rho|}}\prod_{\substack{ % 1 \leq i \leq |\rho| \\
      1 \leq j \leq m
      \\ i\in {\cal A}^\rho_j}}
    H(x_i,y_j)
    \ \prod_{
      \substack{1 \leq k , \ell \le|\rho|
        \\
        \{ k , \ell \}\in E_{\rho_G} 
    }}H(x_\ell,x_k)\mu ( \mathrm{d}x_1 ) \cdots \mu ( \mathrm{d}x_{|\rho|}),
    $$
    and
    \begin{equation}
      \label{cumulant-diagram1}
    \kappa_n\big(N_{y_1,\ldots , y_m}^G\big)=
    \sum_{\substack{\rho\in\Pi_{\widehat{1}} ([n]\times[r])
        \\\rho\wedge\pi=\widehat{0}} \atop {\rm (non-flat \ \! connected)}}
  \lambda^{|\rho |}
  \int_{(\R^d)^{|\rho|}}\prod_{
    \substack{%       1 \leq i \leq |\rho| \\
    1 \leq j \leq m
      \\
    i\in {\cal A}^\rho_j}}  H(x_i,y_j)
  \ \prod_{\substack{
      1\leq k , \ell \le|\rho|
      \\
      \{ k , \ell \}\in E_{\rho_G} }}H(x_\ell,x_k)\mu ( \mathrm{d}x_1)
  \cdots \mu ( \mathrm{d}x_{|\rho|} ).
\end{equation} 
\end{prop}
We note in particular that $N_{y_1,\ldots , y_m}^G$ has positive cumulants, and
the first moment of $N_{y_1,\ldots , y_m}^G$ is given by 
% skewness. For $n=1$, we have 
$$
 \E_\lambda \big[ N_{y_1,\ldots , y_m}^G \big]
% = \kappa_1\big(N_{y_1,\ldots , y_m}^G\big)
 =  \lambda^r 
     \int_{(\R^d)^r}
     \prod_{
       \substack{
         1 \leq i \leq r \\
       1 \leq j \leq m
      \\
      \{v_i,e_j\} \in E_G
     }}  H(x_i,y_j)
  \ \prod_{\substack{
      1\leq k , \ell \leq r 
      \\
      \{ v_k , v_\ell \}\in E_G}}H(x_\ell,x_k)
  \mu ( \mathrm{d}x_1)
  \cdots
  \mu ( \mathrm{d}x_r).
$$ 
\noindent
The cumulant formula of Proposition~\ref{mom-cumfor}
is implemented in the code given in Appendix~\ref{fjkldsf}. 
\section{Subgraph count asymptotics} 
\label{sca}
\noindent
 Next, we investigate the asymptotic behaviour of the cumulants
$\kappa_n\big(N_{y_1,\ldots , y_m}^G\big)$ in \eqref{cumulant-diagram1} as the intensity $\lambda$ tends to infinity. 
 \begin{assumption}
   \label{a1} 
 The connection function 
 $H:\R^d\times \R^d\to[0,1]$
 is translation invariant,
 i.e. $H(x,y) = H(0,y-x)$,
 $x,y \in \real^d$,
 and $\mu$ is the Lebesgue
 measure on $\real^d$, with
 $$
 \int_{\real^d} H(0,y) \mathrm{d}y < \infty.
 $$ 
\end{assumption} 
% \noindent
% When $r=1$ we have $\Pi_{\widehat{1}}[n\times r]=\{\widehat{1}\}$ and $N_{1,1}$ is a Poisson random variable with parameter $\lambda c_H$, so that 
% \begin{equation} \nonumber \kappa_n(N_{1,1})=\lambda \int_{\R^d}H_\lambda(x)\mu ( \mathrm{d}x)  = \lambda c_H, \qquad n \geq 1. \end{equation}
 We recall the following lemma, see \cite[Lemma~2.6]{LiuPrivault}.
\begin{lemma}
% \label{numpartition}
  \noindent
  $a)$ The cardinality of the set \ $\mathcal{C} (n,r)$
 of connected non-flat partitions of $[n]\times[r]$ satisfies 
 \begin{equation}
   \label{coeff-0}
  |\mathcal{C} (n,r) | \leq n!^r r!^{n-1}, 
  \qquad n,r \geq 1. 
\end{equation}
\noindent
$b)$ 
 The cardinality of the set 
$ \mathcal{M}(n,r)$
 of maximal connected non-flat partition of $[n]\times[r]$ satisfies 
 \begin{equation}
   \nonumber % \label{coeff-10}
  |\mathcal{M}(n,r)|=r^{n-1}\prod_{i=1}^{n-1}(1+(r-1)i),
  \qquad n,r\geq 1, 
\end{equation}
 with the bounds 
\begin{equation}\label{coeff-1}
    ( (r-1)r )^{n-1}(n-1)!\leq 
    |\mathcal{M}(n,r)|
     \leq ( (r-1)r )^{n-1}n!, \quad n\geq 1, \ r\geq 2. 
\end{equation}
\end{lemma}
The following result provides
growth estimates for the cumulants of $N^G_{y_1,\ldots , y_m}$. 
\begin{thm}
\label{khopone}
Let $n\geq 1$ and $r\geq 2$, and suppose
that Assumption~\ref{a1} is satisfied.
% and suppose that Assumption~\ref{fjkldsf} is satisfied. 
 We have
  \begin{equation}
\label{onefixed-1}
    0<\kappa_n\big( N_{y_1,\ldots , y_m}^G \big)\leq
    n!^r r!^{n-1}
   (C \lambda )^{1+(r-1)n}
    ,
\end{equation}
and, for $n=2$, 
\begin{equation}
\label{onefixed-2}
  (r-1)r
  c^{2r}
  \lambda^{2r-1} 
  \leq \kappa_2\big(N_{y_1,\ldots , y_m}^G\big)
 \leq 
  r!
  ( C \lambda )^{2r-1}, 
\end{equation}
 where $c, C>0$ are constants independent of $r\geq 2$ and $n \geq 2$.
\end{thm}
\begin{Proof}
 According to Proposition~\ref{mom-cumfor}, every non-flat connected partition $\rho\in\Pi ([n]\times[r])$ corresponds to a summand of order $O(\lambda^{|\rho|
})$. 
% For each $\rho\in\Pi[n\times r]$, we can see from the construction of the diagram $\rho_G$ that merging two nodes into one results into the loss of at most one edge.
 As the maximal cardinality of non-flat connected partitions is
 $1+(r-1)n$,
 the dominating asymptotic order is $O(\lambda^{ 1+(r-1)n })$.
 Precisely, by \eqref{coeff-0}-\eqref{coeff-1} and 
 \eqref{cumulant-diagram1}, letting
 $j_0\in \{1,\ldots , m\}$ such that ${\cal A}^\rho_{j_0} \not= \emptyset$, 
 for some $i_0 \in {\cal A}^\rho_{j_0}$ we have 
\begin{align*} 
 & 
    c^{n|E_G|}
    C^{1+(r-1)n}
    % c_\lambda^{n|E_G|}
    ( (r-1)r )^{n-1}(n-1)!
    \lambda^{1+(r-1)n}
    \\
        & \quad \leq      \kappa_n\big(N_{y_1,\dots,y_m}^G\big)
  \\
   & \quad
    \leq     
    \lambda^{1+(r-1)n}
    \sum_{\substack{\rho\in\Pi_{\widehat{1}} ([n]\times[r])
        \\\rho\wedge\pi=\widehat{0}} \atop {\rm (non-flat \ \! connected) \atop
        % |\rho | = 1 + (r-1)n
    }}
    \int_{(\R^d)^{|\rho|}}\prod_{
      \substack{% 1 \leq i \leq |\rho| \\
        1 \leq j \leq m
      \\
    i\in {\cal A}^\rho_j}} H (x_i,y_j)
  \ \prod_{\substack{
      1\leq k , \ell \le|\rho|
      \\
      \{ k , \ell \}\in E_{\rho_G} }}H (x_\ell,x_k)\mathrm{d}x_1
  \cdots \mathrm{d}x_{|\rho|} 
  \\
   & \quad
    \leq     
    \lambda^{1+(r-1)n}
    \sum_{\substack{\rho\in\Pi_{\widehat{1}} ([n]\times[r])
        \\\rho\wedge\pi=\widehat{0}} \atop {\rm (non-flat \ \! connected) \atop
        % |\rho | = 1 + (r-1)n
    }}
    \int_{(\R^d)^{|\rho|}}
     H (x_{i_0},y_{j_0})
  \ \prod_{\substack{
      1\leq k , \ell \le|\rho|
      \\
      \{ k , \ell \}\in E_{\rho_G} }}H (x_\ell,x_k)\mathrm{d}x_1
  \cdots \mathrm{d}x_{|\rho|} 
  \\
   & \quad
    \leq     
    \lambda^{1+(r-1)n}
    \sum_{\substack{\rho\in\Pi_{\widehat{1}} ([n]\times[r])
        \\\rho\wedge\pi=\widehat{0}} \atop {\rm (non-flat \ \! connected) \atop
        % |\rho | = 1 + (r-1)n
    }}
    \int_{(\R^d)^{|\rho|}}
     H (x_{i_0},y_{j_0})
  \ \prod_{\substack{
      1\leq k , \ell \le|\rho|
      \\
      \{ k , \ell \}\in E_{\rho'_G} }}H (x_\ell,x_k)\mathrm{d}x_1
  \cdots \mathrm{d}x_{|\rho|}, 
\end{align*}
 where for every $\rho \in \Pi_{\widehat{1}} ([n]\times[r])$,
 $\rho'_G$ is a spanning tree contained in $\rho_G$,
 with vertices $\{1,\dots,|\rho|,|\rho|+j\}$
 and such that $|\rho|+j_0$ is a leaf.
 By integrating successively on the variables which
 correspond to leaves of $\rho'_G$ as in the proofs of e.g.
 Theorem~7.1 of \cite{LNS21} or
 Lemma~3.1 of \cite{can2022} and using \eqref{coeff-0}, we obtain 
\begin{align*}
\kappa_n\big(N_{y_1,\dots,y_m}^G\big)\le (C\lambda)^{1+(r-1)n}n!^rr!^{n-1},
\end{align*}
 where $C:= \max \big( 1 , \int_{\real^d} H(0,y) dy \big)$, 
 which yields the right hand side \eqref{onefixed-1}. 
 In addition, Proposition~\ref{mom-cumfor} show
 that all cumulants are positive,
 which completes the proof of \eqref{onefixed-1}. 
 On the other hand, when $r\geq 2$, by
 \eqref{coeff-1} we have 
\begin{equation}
\nonumber
\kappa_2\big(N^G_{y_1,\ldots , y_m}\big)\geq (r-1)rC^{2r} \lambda^{2r-1},
\end{equation}
where $C>0$ is a constant independent of $r \geq 2$ and $n \geq 2$,
which shows \eqref{onefixed-2}. 
\end{Proof}
Theorem~\ref{khopone} also shows the bounds 
\begin{equation}
  \label{lb} 
\frac{C_{r,1}}{\lambda}
\leq
\frac{\kappa_2 \big(N_{y_1,\dots,y_m}^G\big)}{
    \big( \E_\lambda \big[ {N}_{y_1,\ldots , y_m}^G \big] \big)^2
    }
\leq 
\frac{C_{r,2}}{\lambda},
\qquad \lambda > 0, 
\end{equation} 
for some constants $C_{r,1},C_{r,2}>0$ depending only on $r\geq 2$. 
 In what follows, we consider the centered and normalized subgraph count cumulants defined as 
$$
 \widetilde{N}_{y_1,\ldots , y_m}^G
 := \frac{N_{y_1,\ldots , y_m}^G - \kappa_1 \big(N_{y_1,\ldots , y_m}^G \big)}{\sqrt{\kappa_2\big( N_{y_1,\ldots , y_m}^G \big)}}. 
$$

 \begin{corollary}
  \label{jfklds}
  Let $n\geq 2$ and $r\geq 2$. 
% and suppose that Assumption~\ref{fjkldsf} is satisfied. 
 We have 
\begin{equation}
\nonumber % \label{standkop-1}
\big|\kappa_n\big(\widetilde{N}_{y_1,\ldots , y_m}^G\big)\big|\leq n!^r 
C_r^{n/2}
\lambda^{-(n/2-1)},
\end{equation}
  where $C_r > 0$ is a constant depending only on $r \geq 2$.
\end{corollary}
 As a consequence of Corollary~\ref{jfklds}, 
 the skewness of $\widetilde{N}_{y_1,\ldots , y_m}^G$ satisfies  
$$
\big| \kappa_3 \big( \widetilde{N}_{y_1,\ldots , y_m}^G \big)\big|
\leq
C_r \lambda^{-1/2}, 
$$ 
where $C_r > 0$ is a constant depending only on $r \geq 2$.
By Theorem~1 in \cite{Janson1988},
 Corollary~\ref{jfklds} yields the following
 result. 
\begin{prop} 
\label{fjlfa12}
 The renormalized subgraph count $\widetilde{N}_{y_1,\ldots , y_m}^G$ 
 converges in distribution
 to the standard normal distribution
 ${\cal N}(0,1)$ as $\lambda$ tends to infinity.
\end{prop}
 In addition, from Corollary~\ref{jfklds} and
 Lemma~\ref{Statuleviciuscond1}
 the convergence result of Proposition~\ref{fjlfa12} can be
 made more precise via the following 
 convergence bound in the Kolmogorov distance.
 \begin{prop}
   \label{pkol}
   We have 
\begin{equation}
\nonumber
\sup_{x\in\R}\big|\IP_\lambda \big(\widetilde{N}_{y_1,\ldots , y_m}^G \leq x\big)-\Phi(x)\big|\leq C_r \lambda^{ - 1 / ( 4r-2 )},
\qquad  r\geq 2, 
\end{equation}
where $C_r>0$ is a constant depending only on $r\geq 2$
and $\Phi$ is the cumulative distribution function of the standard normal distribution. 
 \end{prop}
  By the second moment method, see e.g. (3.4) page 54 of
\cite{jansongraphs} or Theorem 2.3.2 in \cite{roch},
we also obtain the following lower bound
for endpoint connectivity and subgraph existence. % containment. 
\begin{prop}
  \label{jklf3}
  We have 
\begin{equation}
\label{fjkl21} 
\IP_\lambda \big( 
{N}_{y_1,\ldots , y_m}^G >0 \big) \geq
% \frac{\big( \E_\lambda \big[ {N}_{\{y_1,\ldots , y_m\}}^G \big]\big)^2}{ \kappa_2 \big(N_{y_1,\dots,y_m}^G\big) + \big( \E_\lambda \big[ \big( {N}_{\{y_1,\ldots , y_m\}}^G \big)^2\big]} = 
\frac{\big( \E_\lambda \big[ 
    N_{y_1,\ldots , y_m}^G 
    \big]\big)^2}{ 
  \E_\lambda \big[ \big( N_{y_1,\ldots , y_m}^G \big)^2\big]},
\qquad \lambda >0. 
\end{equation}
\end{prop}
We note from \eqref{lb} that the lower bound
\eqref{fjkl21} converges to $1$ as $\lambda$ tends to infinity. 
\section{Numerical examples} 
\label{examples}
\noindent
% We denote by $0$ the origin in $\R^d$, and consider $x\ne 0$ arbitrary. 
 In this section we assume that $H$ is the Rayleigh connection function
$$
H_\beta (x,y):=e^{-\beta\|x-y\|^2}, \qquad x,y\in \real^d,
$$
where $\beta>0$, and $\mu$ is the Lebesgue measure on $\real^d$.
In this case, Assumption~\ref{a1} is satisfied.
% and we have the relation 
% \begin{equation} \label{fjkl3} \int_{\real^d} \cdots \int_{\real^d} \prod_{i=1}^n H_\beta (x_{i-1},x_i) \mathrm{d}x_1 \cdots \mathrm{d}x_{n-1} = \left( \frac{\pi^{n-1} }{n \beta^{n-1}}\right)^{d/2} H_{\beta / n} (x_0,x_n), \end{equation} see Lemma~5.2 in \cite{prkhp}. 
% and the moment and cumulant formulas of Proposition~\ref{mom-cumfor} become polynomials in $\lambda (\pi / \beta)^{d/2}$. 
 In the following examples, the SageMath code listed in 
 Appendix~\ref{fjkldsf} is run after loading the definitions
 of Table~\ref{t1-00}. 
  
\begin{table}[H] 
  \centering
\scriptsize %   \small
%  \resizebox{\textwidth}{!}
    {
  \begin{tabular}{|ll|ll|} % {\textwidth}{|XX|XX|}
 \hline
 \multicolumn{2}{|l}{
 \EscVerb{load("cumulants_parallel.sage")}
 }
 & \multicolumn{2}{l|}{\# Loading the functions definitions ~~~~~~~~~~~~~~~~~~~~~~~~~~~~~~~~~~~~~~~~~~~~~~~~~~~~~
 }  
 \\
 \hline
 \multicolumn{2}{|l}{
 \EscVerb{λ,β = var("λ,β"); assume(β>0)}
 }
 & \multicolumn{2}{l|}{\# Variable definitions ~~~~~~~~~~~~~~~~~~~~~~~~~~~~~~~~~~~~~~~~~~~~~~~~~~~~~~~~~~~~
 }  
 \\
 \hline
 \multicolumn{2}{|l}{
 \EscVerb{def H(x,y,β): return exp(-β*(x-y)**2)}
 } 
  & \multicolumn{2}{l|}{\# Connection function}  
 \\
 \hline
 \multicolumn{2}{|l}{
 \EscVerb{def mu(x,λ,β): return 1} % λ % *exp(-β*x**2)
}
  & \multicolumn{2}{l|}{\# Flat intensity}   
 \\
\hline
\end{tabular}
}
\caption{Functions definitions.}
\label{t1-00}
\end{table} 

\vspace{-0.4cm}

\noindent
In later code inputs, graphs are denoted by their edge sequence $G$,
and the set of endpoints is denoted by $\EE$. 

\subsection{Three-hop paths with two endpoints}
\noindent
 By a $k$-hop path, we mean a non-self intersecting path having $k$ edges. 
 We take $m=2$, $r=2$, and in Table~\ref{t1} we 
 compute the first three cumulants of $N^G_{y_1,y_2}$ when $G$ is
 a single-edge graph with two endpoints in dimension $d=1$,
 see Figure~\ref{fig3} for an illustration in dimension $2$.  
 Unlike in the two-hop with two endpoints case, this three-hop count does not
 have a Poisson distribution.
% after loading the relevant function definitions. 

\begin{figure}[H]
  \centering
    \includegraphics[width=0.8\linewidth,height=4cm]{../end_points/end_points_multidim/3hops_2endpoints/plot.pdf} 
\caption{Two $3$-hop paths in dimension $d=2$.} 
\label{fig3}
\end{figure}

\vspace{-0.3cm} 

\noindent
 To make cumulant expressions more compact,
 the exact formulas in Table~\ref{t1} 
 are expressed with $y_1=y_2=0$ and $\beta := \pi$,
 in dimension $d=1$. 

\begin{table}[H] 
  \centering
\scriptsize %   \small
 \resizebox{\textwidth}{!}
    {
  \begin{tabular}{|ll|ll|} % {\textwidth}{|XX|XX|}
 \hline
 \multicolumn{2}{|l}{
G = [[1,2]]; $\EE$ =[[1],[2]]; d=1 
 }
 & \multicolumn{2}{l|}{\# Single edge graph $r=2$,
 two endpoint $m=2$, dimension $d=1$~~~~~~~~~~~~~~~~~~~~~~~~~~~~~~~~~~~~~~~~~~~~~~~~~~~~~~~~~~~~~~~}   
 \\
\hline
\end{tabular}
}
  \resizebox{1\textwidth}{!}{
  \begin{tabular}{|ll|ll|} % {\textwidth}{|XX|XX|}
\hline
\multicolumn{1}{|c|}{Command} & \multicolumn{1}{c|}{Order} & \multicolumn{1}{c|}{Cumulant output} & \multicolumn{1}{c|}{Connected non-flat partitions} 
 \\ 
 \hline
\multicolumn{1}{|c|}{c(1,d,G,$\EE$,mu,H)} & \multicolumn{1}{c|}{\normalsize 1st} & \multicolumn{1}{c|}{\large $\frac{1}{\sqrt{3}} {\lambda}^{2}$} & \multicolumn{1}{c|}{\normalsize 1} 
\\
\hline
\multicolumn{1}{|c|}{c(2,d,G,$\EE$,mu,H)} & \multicolumn{1}{c|}{\normalsize 2nd} & \multicolumn{1}{c|}{\large $\left(\frac{1}{\sqrt{3}}+\frac{1}{\sqrt{2}}\right) \lambda ^3+\left(\frac{1}{\sqrt{3}}+\frac{1}{2 \sqrt{2}}\right) \lambda ^2$} & \multicolumn{1}{c|}{\normalsize 6} 
\\
\hline
\multicolumn{1}{|c|}{c(3,d,G,$\EE$,mu,H)} & \multicolumn{1}{c|}{\normalsize 3rd} & \multicolumn{1}{c|}{
  \large $\left(\sqrt{\frac{12}{7}}+\frac{3}{\sqrt{5}}+\frac{3}{\sqrt{7}}+\frac{12}{\sqrt{31}}\right) \lambda ^4+\left(\sqrt{3}+\sqrt{\frac{3}{2}}+\frac{17}{5 \sqrt{2}}+\frac{12}{\sqrt{19}}\right) \lambda^3 +
  \left(
  \frac{3}{2 \sqrt{2}}+\frac{1}{\sqrt{3}}\right)\lambda^2$} & \multicolumn{1}{c|}{\normalsize 68} 
 \\ % [1ex]
\hline
\end{tabular}
}
\caption{Cumulants of the count of $2$-hop paths with two endpoints in dimension $d=1$.}
\label{t1}
\end{table} 

\vspace{-0.4cm}

\noindent
Table~\ref{t1-1} lists the counts of connected non-flat partitions
and runtimes
for the computation of cumulants of orders $1$ to $6$,
and shows that such partitions represent only a fraction (around 25\%) of total partition counts.
Computations in this and the following examples
 are run on a standard desktop computer with
an 8-core CPU at 4.10GHz. 

\begin{table}[H] 
  \centering
\scriptsize %   \small
\resizebox{\textwidth}{!}{
\begin{tabular}{|c||c|c|c|c|c|c||c|c|c|} 
\hline
 Order $n$ & 2 blocks & 3 blocks & 4 blocks & 5 blocks & 6 blocks & 7 blocks & Total & $\Pi ([n]\times[2])$ & Comp. time
\\ 
\hline
 1st & 1 & 0 & 0 & 0 & 0 & 0 & 1 & 2 & 0.5s
\\ 
\hline
 2nd & 2 & 4 & 0 & 0 & 0 & 0 & 6 & 15 & 1s 
\\ 
\hline
 3rd & 4 & 32 & 32 & 0 & 0 & 0 & 68 & 203 & 3s
\\ 
\hline
 4th & 8 & 208 & 624 & 352 & 0 & 0 & 1,192 & 4,140 & 1m
\\ 
\hline
 5th & 16 & 1,280 & 8,960 & 13,904 & 5,040 & 0 & 29,200 & 115,975 & 47m
\\ 
\hline
6th & 32 & 7,744 & 116,160 & 375,776 & 351,456 & 88,544 & 939,712 & 4,213,597 & 29 hours 
\\ 
 \hline
\end{tabular}
}
\caption{Computation times and counts of connected non-flat {\em vs} all
 partitions in $\Pi ([n]\times[2])$.}
\label{t1-1}
\end{table} 

\vspace{-0.4cm}

\noindent
 Figure~\ref{fig2-111} presents connectivity estimates based
 on the moment 
 and cumulant formulas of Propositions~\ref{jklf3} and \ref{fdshkf0},
 in dimension $d=1$. 

\begin{figure}[H]
  \centering
 \begin{subfigure}{0.49\textwidth}
   \includegraphics[width=1\linewidth, height=5cm]{../end_points/end_points_multidim/3hops_2endpoints_dim_1/voidproba} 
    \caption{First and second moment bounds \eqref{fjkl21}.} 
 \end{subfigure}
 \begin{subfigure}{0.49\textwidth}
%   \vskip0.1cm
   \includegraphics[width=1\linewidth, height=5.2cm]{../end_points/end_points_multidim/3hops_2endpoints_dim_1/voidproba_zoom} 
   \vskip-0.1cm
   \caption{Cumulant approximations \eqref{fjkl32} with $n=0$.} 
 \end{subfigure}
 \caption{Connection probabilities.} 
\label{fig2-111} 
\end{figure}

\noindent
\subsection{Four-hop paths with two endpoints}
\noindent 
 Here, we take $m=2$ and $r=3$, and in Table~\ref{t2} we 
 compute the first cumulant of $N^G_{y_1,y_2}$ when $G$ is
 a four-hop path with two endpoints in dimension $d=1$, 
 see Figure~\ref{fig5-1} for an illustration in dimension $2$.  

\begin{figure}[H]
\centering
\includegraphics[width=0.8\linewidth,height=4cm]{../end_points/end_points_multidim/4hops_2endpoints/plot} 
\caption{Four-hop paths with two fixed endpoints in dimension $d=2$.} 
\label{fig5-1}
 \end{figure}

\vspace{-0.3cm} 

\noindent
 The closed-form expressions in Table~\ref{t2} 
 are expressed with $y_1=y_2=0$ and $\beta := \pi$,
 in dimension $d=1$. 

 \begin{table}[H] 
  \centering
% \scriptsize %   \small
\resizebox{\textwidth}{!}
      {
  \begin{tabular}{|ll|ll|} % {\textwidth}{|XX|XX|}
 \hline
 \multicolumn{2}{|l}{
G = [[1,2],[2,3]];  $\EE$=[[1],[3]]; d=1
}
 & \multicolumn{2}{l|}{\# $4$-hop path $r=3$,
 two endpoints $m=2$, dimension $d=1$}   
 \\
\hline
\hline
\multicolumn{1}{|c|}{Instruction} & \multicolumn{1}{c|}{Order} & \multicolumn{1}{c|}{Cumulant output} & \multicolumn{1}{c|}{Connected non-flat partitions} 
 \\ 
 \hline
\multicolumn{1}{|c|}{c(1,d,G,$\EE$,mu,H)} & \multicolumn{1}{c|}{1st} & \multicolumn{1}{c|}{\Large $\frac{{\lambda}^{3}}{2}$} & \multicolumn{1}{c|}{\large 1} 
 \\ 
 \hline
\multicolumn{1}{|c|}{c(2,d,G,$\EE$,mu,H)} & \multicolumn{1}{c|}{2nd} & \multicolumn{1}{c|}{\Large $\frac{1}{6} \lambda ^3 \left(\left(\sqrt{6}+4 \sqrt{\frac{3}{5}}+\frac{3}{2 \sqrt{2}}+\frac{12}{\sqrt{7}}\right) \lambda ^2+\left(3 \sqrt{3}+16 \sqrt{\frac{3}{7}}+8 \sqrt{\frac{3}{11}}+\frac{3}{2 \sqrt{2}}+\frac{6}{\sqrt{5}}\right) \lambda +\sqrt{3}+\sqrt{6}+6\right)$}
        & \multicolumn{1}{c|}{\large 33} 
\\ % [1ex]
\hline
\end{tabular}
}
\caption{First and second cumulants of the count of four-hop paths with two endpoints.}
\label{t2} 
\end{table} 

\vspace{-0.4cm}

%  \medskip
\noindent
Table~\ref{t1-1-2} presents the counts of connected non-flat partitions at different orders, and shows that such partitions represent only a fraction (around 10\%) of total partition counts. 
 
\begin{table}[H] 
  \centering
\scriptsize %   \small
\resizebox{\textwidth}{!}{
\begin{tabular}{|c||c|c|c|c|c|c|c||c|c|c|} 
\hline
 Order $n$ & 3 blocks & 4 blocks & 5 blocks & 6 blocks & 7 blocks & 8 blocks & 9 blocks & Total & $\Pi ([n]\times[3])$ & Comp. time
\\ 
\hline
 1st & 1 & 0 & 0 & 0 & 0 & 0 & 0 & 1 & 5 & 1s
\\ 
\hline
 2nd & 6 & 18 & 9 & 0 & 0 & 0 & 0 & 33 & 203 & 2s 
\\ 
\hline
 3rd & 36 & 540 & 1242 & 864 & 189 & 0 & 0 & 2871 & 21,147 & 4m 
\\ 
\hline
 4th & 216 & 13,608 & 94,284 & 186,624 & 145,908 & 48,276 & 5,589 & 494,500 & 4,213,597 & 19 hours 
\\ 
 \hline
\end{tabular}
}
\caption{Computation times and counts of connected non-flat {\em vs} all partitions in $\Pi ([n]\times[3])$.}
\label{t1-1-2}
\end{table} 

\vspace{-0.4cm}

\noindent
 In Figure~\ref{fig2-11} we plot the corresponding
 moment expressions {\em vs} their Monte Carlo estimates
 in dimension $d=1$ with $y_1 = 0$ and $y_2 = 1$. 

\begin{figure}[H]
  \centering
 \begin{subfigure}[b]{0.49\textwidth}
    \includegraphics[width=1\linewidth, height=4.8cm]{../end_points/end_points_multidim/4hops_2endpoints_dim_1/moment1} 
    \caption{First moment.} 
 \end{subfigure}
 \begin{subfigure}[b]{0.49\textwidth}
    \includegraphics[width=1\linewidth, height=4.8cm]{../end_points/end_points_multidim/4hops_2endpoints_dim_1/moment2} 
    \caption{Second moment.} 
 \end{subfigure}
 \begin{subfigure}[b]{0.50\textwidth}
 \hskip-0.5cm
    \includegraphics[width=1.06\linewidth, height=4.8cm]{../end_points/end_points_multidim/4hops_2endpoints_dim_1/moment3} 
    \caption{Third moment.} 
 \end{subfigure}
  \centering
  \begin{subfigure}[b]{0.49\textwidth}
  \hskip-0.3cm
    \includegraphics[width=1.03\linewidth, height=4.8cm]{../end_points/end_points_multidim/4hops_2endpoints_dim_1/moment4} 
    \caption{Fourth moment.} 
 \end{subfigure}
  \caption{Moment estimates.} 
\label{fig2-11} 
\end{figure}

\subsection{Triangles with endpoints}
 \noindent 
\noindent
 Taking $r=3$ and $m=3$, in Table~\ref{t3} we 
 compute the first and second
 cumulants of $N^G_{y_1,y_1,y_3}$ when $G$ is
 a triangle with three endpoints in dimension $d=1$, 
 see Figure~\ref{fig7} for an illustration in dimension $2$.  

\begin{figure}[H]
\centering
\includegraphics[width=0.8\linewidth,height=4cm]{../end_points/end_points_multidim/three-cycle/plot.png} 
\caption{Triangle with three fixed endpoints in dimension $d=2$.} 
\label{fig7}
\end{figure}

\vspace{-0.3cm} 

\noindent
 The closed-form expressions in Table~\ref{t3} 
 are expressed with $y_1=y_2=y_3=0$ and $\beta := \pi$,
 in dimension $d=1$. 

 \begin{table}[H] 
  \centering
% \scriptsize %   \small
    \resizebox{\textwidth}{!}
      {
  \begin{tabular}{|ll|ll|} % {\textwidth}{|XX|XX|}
 \hline
 \multicolumn{2}{|l}{
   G = [[1,2],[2,3],[3,1]];
    $\EE$=[[1],[2],[3]]; d=1;
}
  & \multicolumn{2}{l|}{\# Triangle graph $r=3$; three endpoints $m=3$; dimension $d=1$} 
 \\
\hline
\hline
\multicolumn{1}{|c|}{Instruction} & \multicolumn{1}{c|}{Order} & \multicolumn{1}{c|}{Cumulant output} & \multicolumn{1}{c|}{Connected non-flat partitions} 
\\ 
\hline
\multicolumn{1}{|c|}{c(1,d,G,$\EE$,mu,H)} & \multicolumn{1}{c|}{1st} & \multicolumn{1}{c|}{\Large $\frac{{\lambda}^{3}}{4}$} & \multicolumn{1}{c|}{1} 
\\ 
\hline
\multicolumn{1}{|c|}{c(2,d,G,$\EE$,mu,H)} & \multicolumn{1}{c|}{2nd} & \multicolumn{1}{c|}{\Large ~~~~$
  \left(
  \frac{\sqrt{3}}{8}   + \frac{3}{8}
  \right)
        {\lambda}^{5}
        +
        \left(
        \frac{2\sqrt{105} }{35}
  + \frac{\sqrt{3} }{5}
  + \frac{3}{4}
  \right) {\lambda}^{4} 
  +
  \left(
  \frac{3\sqrt{35} }{35}
  + \frac{\sqrt{2} }{5}
  + \frac{1}{4}
  \right) {\lambda}^{3}
  $~~~~} & \multicolumn{1}{c|}{33} 
\\ % [1ex]
\hline
\end{tabular}
}
\caption{First and second cumulants of the count of triangles with three endpoints.}
\label{t3}
\end{table} 

\vspace{-0.4cm}

\noindent
Table~\ref{t1-1-3} presents computation times in dimension $d=2$.
 
\vspace{-0.1cm}

\begin{table}[H] 
  \centering
\scriptsize %   \small
\resizebox{\textwidth}{!}{
\begin{tabular}{|c||c|c|c|c|c||c|c|c|} 
\hline
 Order $n$ & 3 blocks & 4 blocks & 5 blocks & 6 blocks & 7 blocks & Total & $\Pi ([n]\times[3])$ & Comp. time
\\ 
\hline
 1st & 1 & 0 & 0 & 0 & 0 & 1 & 5 & 1s
\\ 
\hline
 2nd & 6 & 18 & 9 & 0 & 0 & 33 & 203 & 21s
\\ 
\hline
 3rd & 36 & 540 & 1,242 & 864 & 189 & 2,871 & 21,147 & 1 hour 
\\ 
 \hline
\end{tabular}
}
\caption{Computation times and counts of connected non-flat {\em vs} all partitions in $\Pi ([n]\times[3])$.}
\label{t1-1-3}
\end{table} 

\vspace{-0.4cm}

\noindent
 Figure~\ref{fig5} presents second and third order Gram-Charlier
 expansions \eqref{gram_charlier-1}-\eqref{gram_charlier-2} 
 for the probability density function of the
 count $N^G_{y_1,y_1,y_3}$ of triangles with three endpoints,
 based on exact second and third cumulant expressions. 
 
\begin{figure}[H]
\centering
\begin{subfigure}{.5\textwidth}
\centering
\includegraphics[width=1.\textwidth]{../end_points/end_points_multidim/triangle_3endpoints_50/two_endpoints_50_1.pdf} 
\vskip-0.08cm
\caption{\small $\lambda = 50$.} 
\end{subfigure}
\hskip-0.2cm
\begin{subfigure}{.5\textwidth}
\centering
\includegraphics[width=1.\textwidth]{../end_points/end_points_multidim/triangle_3endpoints_400/two_endpoints_50_1.pdf} 
\vskip-0.08cm
\caption{\small $\lambda = 400$.} 
\end{subfigure}
\vskip-0.16cm
\caption{\small Gram-Charlier density expansions {\em vs} Monte Carlo density estimation.} 
\label{fig5}
\end{figure}

\vspace{-0.3cm}

\noindent
In Figure~\ref{fig5}, the purple areas correspond to probability density estimates obtained by Monte Carlo simulations with $\beta = 2$. 
The second order expansions correspond to
the Gaussian diffusion approximation 
obtained by matching first and second order moments. 
Figure~\ref{fig5} shows that
the actual probability density estimates obtained by simulation 
can be significantly different from
their Gaussian diffusion approximations when 
skewness takes large absolute values. 
In addition, in Figure~\ref{fig5} 
the fourth order Gram-Charlier expansions appear to give the best fit
to the actual probability densities, 
which have positive skewness. 

\subsection{Trees with one endpoint}
\noindent
 Here we take $r=4$ and $m=3$, and in Table~\ref{t3-1} we 
 compute the first and second cumulants of $N^G_{y_1,y_1,y_3}$ when $G$ is
 a tree with three endpoints in dimension $d=2$, 
 see Figure~\ref{fig9} for an illustration. 

\begin{figure}[H]
\centering
\includegraphics[width=0.8\linewidth,height=4cm]{../end_points/end_points_multidim/tree_1endpoint/plot.png} 
\caption{Trees with a single fixed endpoint in dimension $d=2$.} 
\label{fig9}
\end{figure}

\vspace{-0.3cm} 

\begin{table}[H] 
  \centering
% \scriptsize %   \small
    \resizebox{\textwidth}{!}
      {
  \begin{tabular}{|ll|ll|} % {\textwidth}{|XX|XX|}
 \hline
 \multicolumn{2}{|l}{
   G = [[1,2],[2,3],[2,4]];
    $\EE$=[[1,3,4]]; d=2;
}
  & \multicolumn{2}{l|}{\# Tree $r=4$; single endpoint $m=1$; dimension $d=2$} 
 \\
\hline
\hline
\multicolumn{1}{|c|}{Instruction} & \multicolumn{1}{c|}{Order} & \multicolumn{1}{c|}{Cumulant output} & \multicolumn{1}{c|}{Connected non-flat partitions} 
\\ 
\hline
\multicolumn{1}{|c|}{c(1,d,G,$\EE$,mu,H)} & \multicolumn{1}{c|}{\small 1st} & \multicolumn{1}{c|}{\large $\frac{{\lambda}^{4}}{12}$} & \multicolumn{1}{c|}{\small 1} 
\\ 
\hline
\multicolumn{1}{|c|}{c(2,d,G,$\EE$,mu,H)} & \multicolumn{1}{c|}{\small 2nd} & \multicolumn{1}{c|}{\large ~~~~$
\frac{41{\lambda}^{7}}{384}  + \frac{99039{\lambda}^{6}}{165760}  + \frac{232885{\lambda}^{5}}{175824}  + \frac{37{\lambda}^{4}}{50}$~~~~} & \multicolumn{1}{c|}{\small 208} 
\\ % [1ex]
\hline
\end{tabular}
}
\caption{First and second cumulants of the count of trees with one endpoint.}
\label{t3-1}
\end{table} 

\vspace{-0.4cm}

\noindent
The computation times presented in 
Table~\ref{t1-1-4} are for dimension $d=2$.
 
\vspace{-0.2cm}

\begin{table}[H] 
  \centering
\scriptsize %   \small
\resizebox{\textwidth}{!}{
\begin{tabular}{|c||c|c|c|c|c|c|c||c|c|c|} 
\hline
 Order $n$ & 4 blocks & 5 blocks & 6 blocks & 7 blocks & 8 blocks & 9 blocks & 10 blocks & Total & $\Pi ([n]\times[4])$ & Comp. time
\\ 
\hline
 1st & 1 & 0 & 0 & 0 & 0 & 0 & 0 & 1 & 15 & 1s
\\ 
\hline
 2nd & 24 & 96 & 72 & 15 & 0 & 0 & 0 & 208 & 4,140 & 2m
\\ 
\hline
 3rd & 576 & 13,824 & 50,688 & 59,904 & 29,952 & 6,912 & 640 & 162,496 & 4,213,597 & 40 hours
\\ 
 \hline
\end{tabular}
}
\caption{Computation times and counts of connected non-flat {\em vs} all partitions in $\Pi ([n]\times[4])$.} 
\label{t1-1-4}
\end{table} 

\vspace{-0.4cm}

\noindent
In Figure~\ref{fig2-11-2} we plot the
second cumulant of ${N}_{y_1,\ldots , y_m}^G$ 
and the third cumulant of  $\widetilde{N}_{y_1,\ldots , y_m}^G$
 {\em vs} their Monte Carlo estimates
 in dimension $d=2$ with $y_1 = y_2 = y_3 = 0$. 

 \begin{figure}[H]
  \centering
 \begin{subfigure}[b]{0.50\textwidth}
 \hskip-0.5cm
    \includegraphics[width=1.06\linewidth, height=5cm]{../end_points/end_points_multidim/tree_1endpoint/cumulant2} 
    \caption{Second cumulant.} 
 \end{subfigure}
  \centering
  \begin{subfigure}[b]{0.49\textwidth}
  \hskip-0.3cm
    \includegraphics[width=1.03\linewidth, height=5cm]{../end_points/end_points_multidim/tree_1endpoint/normalizedcumulant3} 
    \caption{Normalized third cumulant.} 
 \end{subfigure}
\caption{Cumulant estimates.} 
\label{fig2-11-2} 
\end{figure}

\subsection{Correlation of triangles {\em vs} four-hop counts} 
\noindent
In this example, we run the joint cumulant code provided
in Appendix~\ref{fjkldsf-2}
to compute the correlation of triangle and four-hop counts
without endpoints, as a function of the intensity parameter $\lambda$.
 Here, $\mu$ is taken to be a finite measure
 as no endpoints are considered here, i.e.
 we have $\EE$=[ \hskip0.03cm ] and $m=0$,
 and the SageMath code listed in 
 Appendix~\ref{fjkldsf} is run after loading the definitions
 of Table~\ref{t1-002}. 
 
\begin{table}[H] 
  \centering
\scriptsize %   \small
 \resizebox{\textwidth}{!}
    {
  \begin{tabular}{|ll|ll|} % {\textwidth}{|XX|XX|}
 \hline
 \multicolumn{2}{|l}{
 \EscVerb{load("cumulants_parallel.sage");load("jointcumulants.sage")}
 }
 & \multicolumn{2}{l|}{\# Loading the functions definitions ~~~~~~~~~~
 }  
 \\
 \hline
 \multicolumn{2}{|l}{
 \EscVerb{λ,β = var("λ,β"); assume(β>0)}
 }
 & \multicolumn{2}{l|}{\# Variable definitions ~~~~~~~~~~~~~~~~~~~
 }  
 \\
 \hline
 \multicolumn{2}{|l}{
 \EscVerb{def H(x,y,β): return exp(-β*(x-y)**2)}
 } 
  & \multicolumn{2}{l|}{\# Connection function}  
 \\
 \hline
 \multicolumn{2}{|l}{
 \EscVerb{def mu(x,λ,β): return exp(-β*x**2)} % λ % *exp(-β*x**2)
}
  & \multicolumn{2}{l|}{\# Finite intensity measure}   
 \\
\hline
\end{tabular}
}
\caption{Functions definitions.}
\label{t1-002}
\end{table} 

\vspace{-0.4cm}

\noindent
 The closed-form expressions in Table~\ref{t3-1-1}
 are expressed with $\beta := \pi$, in dimension $d=2$. 

\begin{table}[H] 
  \centering
% \scriptsize %   \small
    \resizebox{1.0\textwidth}{!}
      {
  \begin{tabular}{|ll|ll|} % {\textwidth}{|XX|XX|}
 \hline
 \multicolumn{4}{|l|}{
G1 = [[1,2],[2,3],[3,1]]; 
G2 = [[1,2],[2,3],[3,4],[4,5]]; 
G2c = [[4,5],[5,6],[6,7],[7,8]]; 
G = [G1,G2c]; $\EE$=[]; d=2;
}
 \\
 \hline
 \multicolumn{4}{|l|}{\# Triangles G1 and $4$-hops G2;
    $r_1=3$, $r_2=5$; no endpoints $m=0$; dimension $d=2$} 
 \\
\hline
\hline
\multicolumn{1}{|c|}{Instruction} & \multicolumn{1}{c|}{Order} & \multicolumn{1}{c|}{Cumulant output} & \multicolumn{1}{c|}{Connected non-flat partitions} 
\\ 
\hline
\multicolumn{1}{|c|}{c(2,d,G1,$\EE$,mu,H)} & \multicolumn{1}{c|}{\small 2nd} & \multicolumn{1}{c|}{\normalsize \Large$\frac{3 \lambda^5}{64} + \frac{6\lambda^4}{25} + \frac{3\lambda^3}{8}$} & \multicolumn{1}{c|}{\small 33} 
\\ 
\hline
\multicolumn{1}{|c|}{c(2,d,G2,$\EE$,mu,H)} & \multicolumn{1}{c|}{\small 2nd} & \multicolumn{1}{c|}{\Large$
  \frac{7344738590701\lambda^9}{687218605505250} + \cdots 
    $} & \multicolumn{1}{c|}{\small 1545} 
\\ 
\hline
\multicolumn{1}{|c|}{jc(d,G,$\EE$,mu,H)} & \multicolumn{1}{c|}{\small 2nd joint} & \multicolumn{1}{c|}{\Large ~~~~~~$
  \frac{34409 \lambda^7}{1537920} +
  \frac{9101145477 \lambda^6}{55004486680}
  + \frac{10774977 \lambda^5}{28148120}
  $~~~~~} & \multicolumn{1}{c|}{\small 135} 
\\ % [1ex]
\hline
\end{tabular}
}
\caption{Second (joint) moments of triangle counts {\em vs} four-hop counts}\label{t3-1-1}
\end{table} 

\vspace{-0.4cm}

% We note that $$ \left( \frac{34409 }{1537920}\right)^2 =\frac{3}{64} \times \frac{7344738590701}{687218605505250} $$ 
\noindent
In Figure~\ref{fig2-11-3} we plot the 
second joint cumulant and correlation of $\big({N}^{G_1},{N}^{G_2}\big)$ 
 {\em vs} their Monte Carlo estimates
 in dimension $d=1$. 

 \begin{figure}[H]
  \centering
 \begin{subfigure}[b]{0.50\textwidth}
 \hskip-0.5cm
    \includegraphics[width=1.06\linewidth, height=5cm]{../end_points/end_points_multidim/triangles_vs_4_hops_correl/moment12.pdf} 
    \caption{Second joint cumulant.} 
 \end{subfigure}
  \centering
  \begin{subfigure}[b]{0.49\textwidth}
  \hskip-0.3cm
    \includegraphics[width=1.03\linewidth, height=5cm]{../end_points/end_points_multidim/triangles_vs_4_hops_correl/correlation_triangle_vs_4_hops.pdf} 
    \caption{Correlation.} 
 \end{subfigure}
\caption{Correlation and second joint cumulant estimates.} 
\label{fig2-11-3} 
\end{figure}
\noindent
The limit correlation as $\lambda$ tends to infinity can be
exactly estimated from Table~\ref{t3-1-1} as 
$$
\frac{34409 }{1537920} \sqrt{\frac{64}{3} \times
  \frac{687218605505250}{7344738590701}} \approx 0.999602.
$$ 

\appendix

\section{Multivariate moment and cumulant formulae}
\label{appendixa}
\noindent 
In this section, we prove an extension of Proposition~\ref{mom-cumfor}
for the joint moments and cumulants of subgraph counts. 
The next definition extends Definition~\ref{def-1}.
\begin{definition}
Given $r_1,\dots,r_n \geq 1$, we set 
$$
  \pi_i=\left\{(i,1), \ldots ,(i,r_i)\right\},
  \quad
  i=1, \ldots , n, 
$$
 and $\pi := \{ \pi_1,\ldots , \pi_n \}$. 
\begin{enumerate}[i)]
   \item A set partition $\sigma\in\Pi( \pi_1 \cup \cdots \cup \pi_n )$ is connected if $\sigma\vee\pi=\widehat{1}$.      
\item 
 A set partition $\sigma\in\Pi ( \pi_1 \cup \cdots \cup \pi_n )$ is non-flat if $\sigma\wedge\pi=\widehat{0}$. 
\end{enumerate} 
\noindent
 We let $\Pi_{\widehat{1}}( \pi_1 \cup \cdots \cup \pi_n )$ denote the collection of all connected partitions of $\pi_1 \cup \cdots \cup \pi_n$. 
\end{definition}
  In what follows, every partition
  $\rho \in \Pi(\pi_1\cup \cdots \cup \pi_n )$
  will be arranged into a 
 diagram denoted by $\Gamma(\rho ,\pi)$, 
 by arranging $\pi_1,\dots,\pi_n$ into $n$ rows 
 and connecting together the elements of every block of $\rho$, 
 see 
 Figure~\ref{fig:diagram2} for two illustrations with
 $n=5$, 
 $(r_1, r_2 , r_3 , r_4 , r_5) = (3,2,4,3,4)$. 
% and $N=16$. 
 
\begin{figure}[H]
\captionsetup[subfigure]{font=footnotesize}
\centering
\subcaptionbox{Non-connected partition diagram $\Gamma(\rho,\pi)$.}[.5\textwidth]{%
\begin{tikzpicture}[scale=0.9] 
\draw[black, thick] (0,0) rectangle (5,6);

\node[anchor=east,font=\small] at (0.8,5) {1};
\node[anchor=east,font=\small] at (0.8,4) {2};
\node[anchor=east,font=\small] at (0.8,3) {3};
\node[anchor=east,font=\small] at (0.8,2) {4};
\node[anchor=east,font=\small] at (0.8,1) {5};

\node[anchor=south,font=\small] at (1,0) {1};
\node[anchor=south,font=\small] at (2,0) {2};
\node[anchor=south,font=\small] at (3,0) {3};
\node[anchor=south,font=\small] at (4,0) {4};

\filldraw [gray] (1,1) circle (2pt);
\filldraw [gray] (2,1) circle (2pt);
\filldraw [gray] (3,1) circle (2pt);
\filldraw [gray] (4,1) circle (2pt);
\filldraw [gray] (1,2) circle (2pt);
\filldraw [gray] (2,2) circle (2pt);
\filldraw [gray] (3,2) circle (2pt);
%\filldraw [gray] (4,2) circle (2pt);
\filldraw [gray] (1,3) circle (2pt);
\filldraw [gray] (2,3) circle (2pt);
\filldraw [gray] (3,3) circle (2pt);
\filldraw [gray] (4,3) circle (2pt);
\filldraw [gray] (2,3) circle (2pt);
\filldraw [gray] (1,4) circle (2pt);
\filldraw [gray] (2,4) circle (2pt);
%\filldraw [gray] (3,4) circle (2pt);
%\filldraw [gray] (4,4) circle (2pt);
\filldraw [gray] (1,5) circle (2pt);
\filldraw [gray] (2,5) circle (2pt);
\filldraw [gray] (3,5) circle (2pt);
%\filldraw [gray] (4,5) circle (2pt);

\draw[very thick] (1,5) -- (1,4);
\draw[very thick] (2,5) -- (2,4);
\draw[very thick] (2,5) -- (3,5);

\draw[very thick] (1,2) -- (1,1);
\draw[very thick] (2,3) -- (2,2);
\draw[very thick] (2,1) -- (3,1) -- (4,1);
\draw[very thick] (3,2) -- (4,3);

\end{tikzpicture}}%
\subcaptionbox{Connected partition diagram $\Gamma(\rho,\pi)$.}[.5\textwidth]{
\begin{tikzpicture}[scale=0.9] 
\draw[black, thick] (0,0) rectangle (5,6);

\node[anchor=east,font=\small] at (0.8,5) {1};
\node[anchor=east,font=\small] at (0.8,4) {2};
\node[anchor=east,font=\small] at (0.8,3) {3};
\node[anchor=east,font=\small] at (0.8,2) {4};
\node[anchor=east,font=\small] at (0.8,1) {5};

\node[anchor=south,font=\small] at (1,0) {1};
\node[anchor=south,font=\small] at (2,0) {2};
\node[anchor=south,font=\small] at (3,0) {3};
\node[anchor=south,font=\small] at (4,0) {4};

\filldraw [gray] (1,1) circle (2pt);
\filldraw [gray] (2,1) circle (2pt);
\filldraw [gray] (3,1) circle (2pt);
\filldraw [gray] (4,1) circle (2pt);
\filldraw [gray] (1,2) circle (2pt);
\filldraw [gray] (2,2) circle (2pt);
\filldraw [gray] (3,2) circle (2pt);
%\filldraw [gray] (4,2) circle (2pt);
\filldraw [gray] (1,3) circle (2pt);
\filldraw [gray] (2,3) circle (2pt);
\filldraw [gray] (3,3) circle (2pt);
\filldraw [gray] (4,3) circle (2pt);
\filldraw [gray] (2,3) circle (2pt);
\filldraw [gray] (1,4) circle (2pt);
\filldraw [gray] (2,4) circle (2pt);
%\filldraw [gray] (3,4) circle (2pt);
%\filldraw [gray] (4,4) circle (2pt);
\filldraw [gray] (1,5) circle (2pt);
\filldraw [gray] (2,5) circle (2pt);
\filldraw [gray] (3,5) circle (2pt);
%\filldraw [gray] (4,5) circle (2pt);

\draw[very thick] (1,5) -- (1,4); 
%\draw[very thick] (3,5) -- (4,4);

\draw[very thick] (1,2) -- (1,1);
\draw[very thick] (2,2) -- (2,4);
\draw[very thick] (2,1) -- (3,2) -- (4,3);

\end{tikzpicture}}%
\caption{Two examples of partition diagrams.}
\label{fig:diagram2}
\end{figure}
\vspace{-0.4cm}

% It is intuitively clear that each diagram $\Gamma(\rho,\pi)$ can be uniquely decomposed into connected sub-diagrams.
\noindent
 Definition~\ref{part-1} extends \cite[Definition~2.4]{LiuPrivault}
to the multivariate setting. 
\begin{definition}\label{part-1}
  % Let $n\ge1$, $r_1,\dots,r_n\in\N$, and $N:=\sum_{i=1}^nr_i$.
  ~~
  \begin{enumerate}[\rm 1)]
\item
  Given $\rho\in\Pi(\pi_1\cup \cdots \cup \pi_n )$,
  we let $\sigma_\rho$ be the partition of $[n]$ defined by
  the condition 
  $$\rho\vee\pi=\bigg\{\bigcup_{i\in b}\pi_i:b\in\sigma_\rho \bigg\}. 
$$
\item For any non-empty set $b\subseteq[n]$, we let % $\rho_b\subseteq\rho$ be defined as 
$$\rho_b:=\bigg\{c\in\rho:c\subseteq\bigcup_{i\in b}\pi_i\bigg\}.
$$
\end{enumerate}
\end{definition}
 As an example, in Figure~\ref{fig:diagram3}-$a)$, when $b = \{1,2\}$ we have
$$
\rho_{\{1,2\}} = \big\{\{(1,1),(2,1)\}, \{(1,2),(1,3),(2,2)\}\big\}. 
$$
% \medskip

\tikzset{hide labels/.style={every label/.append style={text opacity=0}}}

\begin{figure}[H]
\captionsetup[subfigure]{font=footnotesize}
\centering
\subcaptionbox{Connected subpartition $\rho_{\{1,2\}}$.}[.5\textwidth]{%
\begin{tikzpicture}[hide labels, scale=0.9]
\tikzstyle{VertexStyle}=[shape = circle, fill = blue!20, minimum size = 0pt, scale=0., text = white, hide labels]
\draw[black, thick] (0,0) rectangle (5,6);
\node[anchor=east,font=\small] at (0.8,5) {1};
\node[anchor=east,font=\small] at (0.8,4) {2};
\node[anchor=east,font=\small] at (0.8,3) {3};
\node[anchor=east,font=\small] at (0.8,2) {4};
\node[anchor=east,font=\small] at (0.8,1) {5};
\node[anchor=south,font=\small] at (1,0) {1};
\node[anchor=south,font=\small] at (2,0) {2};
\node[anchor=south,font=\small] at (3,0) {3};
\node[anchor=south,font=\small] at (4,0) {4};
\filldraw [gray] (1,1) circle (2pt);
\filldraw [gray] (2,1) circle (2pt);
\filldraw [gray] (3,1) circle (2pt);
\filldraw [gray] (4,1) circle (2pt);
\filldraw [gray] (1,2) circle (2pt);
\filldraw [gray] (2,2) circle (2pt);
\filldraw [gray] (3,2) circle (2pt);
%\filldraw [gray] (4,2) circle (2pt);
\filldraw [gray] (1,3) circle (2pt);
\filldraw [gray] (2,3) circle (2pt);
\filldraw [gray] (3,3) circle (2pt);
\filldraw [gray] (4,3) circle (2pt);
\filldraw [gray] (2,3) circle (2pt);
\filldraw [gray] (1,4) circle (2pt);
\filldraw [gray] (2,4) circle (2pt);
%\filldraw [gray] (3,4) circle (2pt);
%\filldraw [gray] (4,4) circle (2pt);
\filldraw [gray] (1,5) circle (2pt);
\filldraw [gray] (2,5) circle (2pt);
\filldraw [gray] (3,5) circle (2pt);
%\filldraw [gray] (4,5) circle (2pt);
\draw[very thick] (1,5) -- (1,4);
\draw[very thick] (2,5) -- (2,4);
\draw[very thick] (2,5) -- (3,5);
\draw[very thick] (1,2) -- (1,1);
\draw[very thick] (2,3) -- (2,2);
\draw[very thick] (2,1) -- (3,1) -- (4,1);
\draw[very thick] (3,2) -- (4,3);
\node (1) [label=above:{}] at (1,5) {};
\node (2) [label=above:{}] at (2,5) {};
\node (3) [label=above:{}] at (3,5) {};
\node (4) [label=above:{}] at (4,5) {};
\node (5) [label=above:{}] at (1,4) {};
\node (6) [label=above:{}] at (2,4) {};
\node (7) [label=above:{}] at (3,4) {};
\node (8) [label=above:{}] at (4,4) {};
\draw[very thick,blue] \convexpath{1,3,7,5}{.2cm};
\draw[blue,line width=1mm, ->] node[font=\fontsize{12}{0}\selectfont, right=of 2, right=1.5cm, below=-0.9cm] {$~~~~~~\rho_{\{1,2\}}$};
\end{tikzpicture}}%
\subcaptionbox{Splitting $\rho$ into connected subpartitions $\rho_{b_1}$, $\rho_{b_2}$.}[.5\textwidth]{%
\begin{tikzpicture}[scale=0.9] 
\draw[black, thick] (0,0) rectangle (5,6);
\node[anchor=east,font=\small] at (0.8,5) {1};
\node[anchor=east,font=\small] at (0.8,4) {2};
\node[anchor=east,font=\small] at (0.8,3) {3};
\node[anchor=east,font=\small] at (0.8,2) {4};
\node[anchor=east,font=\small] at (0.8,1) {5};
\node[anchor=south,font=\small] at (1,0) {1};
\node[anchor=south,font=\small] at (2,0) {2};
\node[anchor=south,font=\small] at (3,0) {3};
\node[anchor=south,font=\small] at (4,0) {4};
\filldraw [gray] (1,1) circle (2pt);
\filldraw [gray] (2,1) circle (2pt);
\filldraw [gray] (3,1) circle (2pt);
\filldraw [gray] (4,1) circle (2pt);
\filldraw [gray] (1,2) circle (2pt);
\filldraw [gray] (2,2) circle (2pt);
\filldraw [gray] (3,2) circle (2pt);
%\filldraw [gray] (4,2) circle (2pt);
\filldraw [gray] (1,3) circle (2pt);
\filldraw [gray] (2,3) circle (2pt);
\filldraw [gray] (3,3) circle (2pt);
\filldraw [gray] (4,3) circle (2pt);
\filldraw [gray] (2,3) circle (2pt);
\filldraw [gray] (1,4) circle (2pt);
\filldraw [gray] (2,4) circle (2pt);
%\filldraw [gray] (3,4) circle (2pt);
%\filldraw [gray] (4,4) circle (2pt);
\filldraw [gray] (1,5) circle (2pt);
\filldraw [gray] (2,5) circle (2pt);
\filldraw [gray] (3,5) circle (2pt);
%\filldraw [gray] (4,5) circle (2pt);
\draw[very thick] (1,5) -- (1,4);
\draw[very thick] (2,5) -- (2,4);
\draw[very thick] (2,5) -- (3,5);
\draw[very thick] (1,2) -- (1,1);
\draw[very thick] (2,3) -- (2,2);
\draw[very thick] (2,1) -- (3,1) -- (4,1);
\draw[very thick] (3,2) -- (4,3);
\node (1) [label=above:{}] at (1,5) {};
\node (2) [label=above:{}] at (2,5) {};
\node (3) [label=above:{}] at (3,5) {};
\node (4) [label=above:{}] at (4,3) {};
\node (5) [label=above:{}] at (1,4) {};
\node (6) [label=above:{}] at (1,3) {};
\node (7) [label=above:{}] at (3,4) {};
\node (8) [label=above:{}] at (1,1) {};
\node (9) [label=above:{}] at (4,1) {};
\node (10) [label=above:{}] at (4,2) {};
\node (11) [label=above:{}] at (4,5) {};
\draw[very thick,blue] \convexpath{1,3,7,5}{.2cm};
\draw[blue,line width=1mm, ->] node[font=\fontsize{12}{0}\selectfont, right=of 11, right=1.5cm, below=0 cm] {$~~~~~~\rho_{b_1}$};
\draw[very thick,blue] \convexpath{6,4,9,8}{.2cm};
\draw[blue,line width=1mm, ->] node[font=\fontsize{12}{0}\selectfont, right=of 10, left=-.5cm, below=0 cm] {$~~~~~~~\rho_{b_2}$};
\end{tikzpicture}}%
\caption{Diagram $\Gamma(\rho,\pi)$ and splitting of the partition $\rho$ with $\rho\vee\pi=\{\pi_1\cup\pi_2,\pi_3\cup\pi_4\cup\pi_5\}$.}
\label{fig:diagram3}
\end{figure}

\vspace{-0.3cm}

\noindent 
 We note that for $b\subseteq[n]$ we have $\pi_b=\{\pi_i:i\in b\}$, 
 and any partition $\rho \in \Pi ( \pi_1\cup \cdots \cup \pi_n )$ 
 can be split into subpartitions deduced
 from the connected components of $\Gamma(\rho,\pi)$, i.e. 
 \begin{equation}
\nonumber %    \label{fjklds1}
   \rho=\bigcup_{b\in\sigma_\rho}\rho_b, 
\end{equation} 
 as illustrated in Figure~\ref{fig:diagram3}-$b)$ with
 $b_1 = \{ 1,2\}$, $b_2 = \{3,4,5\}$,
 and $\sigma_\rho = \{ b_1,b_2\}$. 
\begin{definition}
% \label{jklf3.1}
 For $\sigma\in\Pi([n])$ 
 we let $\Pi_{\sigma}(\pi_1\cup \cdots \cup \pi_n )$ 
 denote the collection of partitions
$\rho\in\Pi(\pi_1\cup \cdots \cup \pi_n)$ such that
$$\rho\vee\pi=\bigg\{\bigcup_{i\in b}\pi_i:b\in\sigma\bigg\}.
$$
\end{definition} 
In particular,
$\Pi_{\widehat{1}}(\pi_1\cup \cdots \cup \pi_n )$
represents the
set of connected partitions of 
$\pi_1\cup \cdots \cup \pi_n $,
and
$\Pi_{\widehat{0}}(\pi_1\cup \cdots \cup \pi_n )$
represents the partitions of 
$\pi_1\cup \cdots \cup \pi_n )$
that are finer than
$\pi := \{ \pi_1,\ldots , \pi_n \}$.

\medskip
 
% Definition~\ref{jklf3.1} allows us to partition $\Pi(\pi_1\cup \cdots \cup \pi_n )$ as $$\Pi(\pi_1\cup \cdots \cup \pi_n )=\bigcup_{\sigma\in\Pi([n])}\Pi_\sigma(\pi_1\cup \cdots \cup \pi_n ). $$
% \begin{definition} A partition $\rho \in \Pi (\pi_1\cup \cdots \cup \pi_n )$ is said to be {\it non-flat} if $\rho\wedge\pi=\widehat{0}$, and {\it connected} if $\rho\in\Pi_{\widehat{1}}(\pi_1\cup \cdots \cup \pi_n )$. \end{definition} 
 Given $F:\Pi'(\pi_1\cup \cdots \cup \pi_n )\to\R$, % a virtual field on $[n]$,
 where $\Pi'(\pi_1\cup \cdots \cup \pi_n )$ is the collection of all subpartitions of $\pi_1\cup \cdots \cup \pi_n $, 
 we define the mixed moments $\widehat{F}:2^{[n]}\to\R$ by 
\begin{equation}
\label{mm} 
  \widehat{F}(A)=\sum_{\rho\in\Pi(\cup_{i\in A}\pi_i)}F(\rho),
  \qquad
   A\subseteq[n], 
\end{equation}
cf. \cite[p.~33]{MalyshevMinlos91}.
The semi-invariants $C_F:2^{[n]}\to\R$ are defined by the induction formula
 $C_F(A)=\widehat{F}(A)$ when $|A|=1$, and 
\begin{equation}
\nonumber
C_F(A)=\widehat{F}(A)-\sum_{\substack{\{b_1,\dots,b_k\}\in\Pi(A)\\k\ge2}}\prod_{i=1}^kC_F(b_i),
\end{equation}
for $|A|>1$,
see Relation~(16) page~33 of \cite{MalyshevMinlos91}, 
where the sum is taken over all partitions $\sigma\in\Pi(A)$
such that $|\sigma|\ge2$,
i.e.
\begin{equation}
\label{2} 
C_F(A)=\sum_{\rho\in\Pi(A)}
  (-1)^{|\rho|} (|\rho|-1)!
 \prod_{b\in\rho} \widehat{F}( b),
\end{equation}
 see Relation~(16') in \cite{MalyshevMinlos91}.
 The next proposition generalizes \cite[Proposition~3.3]{LiuPrivault} to
the multivariate case.
\begin{prop}
% \label{fact-02}
 Suppose that $F$ satisfies the connectedness factorization property 
\begin{equation}\label{factor-1}
  F(\rho)=\prod_{b\in\sigma_\rho}F(\rho_b),
  \qquad
   \rho\in\Pi'(\pi_1\cup \cdots \cup \pi_n ). 
\end{equation}
 Then, the semi-invariants are given by
 \begin{equation}
   \label{eq-1} 
  C_F(A)=\sum_{\rho\in\Pi_{\widehat{1}}(\cup_{i\in A}\pi_i)}F(\rho),
  \qquad
  \emptyset \not= A\subseteq [n].
\end{equation}
\end{prop}
\begin{Proof}
 \noindent
 $(i)$ 
 It is clear that \eqref{eq-1} holds when $|A|=1$. 
 When $|A|=2$, taking $A=\{i,j\} \subseteq[n]$,
 $i\not= j$, we have 
\begin{eqnarray*}
C_F(A)&=&\widehat{F}(\{i,j\})-C_F(\{i\})C_F(\{j\})\\
&=&\sum_{\rho\in\Pi(\pi_i \cup\pi_j )}F(\rho)-\widehat{F}(\{i\})\widehat{F}(\{j\})\\
&=&\sum_{\rho\in\Pi_{\widehat{1}}(\pi_i \cup\pi_j )}F(\rho)+\sum_{\rho\in\Pi_{\widehat{0}}(\pi_i \cup\pi_j )}F(\rho)-\left(
\sum_{\rho_1\in\Pi(\pi_i)}F(\rho_1)\right)
\left(
\sum_{\rho_2\in\Pi(\pi_j)}F(\rho_2)\right).
\end{eqnarray*}
By splitting any $\rho\in\Pi_{\widehat{0}}(\pi_i\cup\pi_j)$
into two disjoint subpartitions according to Definition~\ref{part-1}, i.e. 
$$\rho=\rho_{\{i\}}\cup\rho_{\{j\}}, 
$$
together with the factorization property \eqref{factor-1}, we find 
\begin{eqnarray}
  \nonumber
  \sum_{\rho\in\Pi_{\widehat{0}}(\pi_i \cup\pi_j)}F(\rho)&=&\sum_{\substack{\rho\in\Pi_{\widehat{0}}(\pi_i\cup\pi_j)
      \\
\nonumber
\rho=\rho_{\{i\}}\cup\rho_{\{j\}}}}F(\rho_{\{i\}})F(\rho_{\{j\}})
  \\
\nonumber
  &=&\left(
\sum_{\rho_1\in\Pi(\pi_i)}F(\rho_1)\right)
\left(
\sum_{\rho_2\in\Pi(\pi_j)}F(\rho_2)\right), 
\end{eqnarray}
 which shows \eqref{eq-1}. 

 \smallskip

 \noindent
 $(ii)$ 
 Next, suppose that \eqref{eq-1} holds for any $A\subseteq[n]$ with $|A|\le m\le n-1$. Let $A\subseteq[n]$ be a subset of $[n]$ with $|A|=m+1$. We have 
\begin{eqnarray*}
  \widehat{F}(A)&=&
  % \sum_{\rho\in\Pi_{\widehat{1}}(\cup_{i\in A}\pi_i)}F(\rho)+
  \sum_{\substack{\rho\in\Pi(\cup_{i\in A}\pi_i)
      %      \\      |\rho\vee\pi_A|\ge 1
    }
    }F(\rho)
  \\
  &=&% \sum_{\rho\in\Pi_{\widehat{1}}(\cup_{i\in A}\pi_i)}F(\rho)+
  \sum_{\substack{\sigma=\{b_1,\dots,b_k\}\in\Pi(A)\\k\ge 1}}\sum_{\substack{\rho\in\Pi(\cup_{i\in A}\pi_i)\\
\rho\vee\pi_A=\{\cup_{i\in b_j}\pi_i\}_{j=1}^k}}F(\rho)\\
  &=&% \sum_{\rho\in\Pi_{\widehat{1}}(\cup_{i\in A}\pi_i)}F(\rho)
  \sum_{\substack{\sigma=\{b_1,\dots,b_k\}\in\Pi(A)\\k\ge 1}}\sum_{\substack{\rho\in\Pi(\cup_{i\in A}\pi_i)\\
\rho\vee\pi_A=\{\cup_{i\in b_j}\pi_i\}_{j=1}^k}}\prod_{j=1}^kF(\rho_{b_j})\\
  &=&% \sum_{\rho\in\Pi_{\widehat{1}}(\cup_{i\in A}\pi_i)}F(\rho)+
  \sum_{\substack{\sigma=\{b_1,\dots,b_k\}\in\Pi(A)\\k\ge 1}}\prod_{j=1}^k
 \sum_{\substack{\rho_j\in\Pi(\cup_{i\in b_j}\pi_i)\\
     \rho_j\vee\pi_{b_j}=\widehat{1}}}F(\rho_{b_i})
 \\
 &=&% \sum_{\rho\in\Pi_{\widehat{1}}(\cup_{i\in A}\pi_i)}F(\rho)
 \sum_{\substack{\sigma=\{b_1,\dots,b_k\}\in\Pi(A)\\k\ge 1}}\prod_{j=1}^k
\sum_{\rho_j\in\Pi_{\widehat{1}}(\cup_{i\in b_j}\pi_i)}F(\rho_{b_i})
\\
&=& \sum_{\rho\in\Pi_{\widehat{1}}(\cup_{i\in A}\pi_i)}F(\rho)
+ \sum_{\substack{\{b_1,\dots,b_k\}\in\Pi(A)\\k\ge 2}}\prod_{j=1}^kC_F(b_j),
\end{eqnarray*}
where the last equality follows from the induction hypothesis \eqref{eq-1} when $|A|\le m$. The proof is completed by subtracting the last term from both sides.
\end{Proof}
Given $n\ge1$ and $f^{(i)}:(\R^d)^{r_i}\to\R$, $i=1,\dots,n$,
 measurable functions, we let 
$$
 \left(\bigotimes_{i=1}^nf^{(i)} \right)(x_{1,1},\ldots,x_{1,r_1},\ldots,
 x_{n,1},\ldots , x_{n,r_n}):=\prod_{i=1}^nf^{(i)} (x_{i,1},\dots,x_{i,r_i}). 
% \qquad x_1,\dots,x_N\in \R^d.
  $$
  For $\rho\in \Pi(\pi_1\cup \cdots \cup \pi_n )$, we also denote by
  $\big(
  \bigotimes_{i=1}^nf^{(i)} \big)_{\hskip-0.01cm \rho}:(\R^d)^{|\rho|}\to\R$ the function
  obtained by equating any two variables
  whose indexes belong to a same block of $\rho$.
  We refer to \cite[Theorem~3.1]{bogdan}
  for the next result.
  %   which generalizes \cite[Proposition~3]{prkhp} to the multivariate case.
  \begin{prop}
    \label{moment-1}
    Let $n\ge1$,
    $r_1,\ldots ,r_n\ge1$,
     and let $f^{(i)} :(\R^d)^{r_i}\to\R$ be a measurable function 
    for $i=1,\dots,n$. We have
\begin{equation}
\nonumber
\E\left[\prod_{i=1}^n
 \int_{(\R^d)^{r_i}}f^{(i)} (z_1,\dots,z_{r_i})
  \ \eta(\mathrm{d}z_1)\cdots\eta(\mathrm{d}z_{r_i}) 
\right]=\sum_{\rho\in\Pi(\pi_1\cup \cdots \cup \pi_n )}\lambda^{|\rho|}\int_{(\R^d)^{|\rho|}}\left(\bigotimes_{i=1}^n f^{(i)} \right)_{\hskip-0.1cm \rho}(\mathbf{x})\mathrm{d}\mathbf{x},
\end{equation}
% and $N:=\sum_{i=1}^nr_i$. 
\end{prop}
Proposition~\ref{moment-1} can be specialized as follows.
\begin{corollary}\label{moment-2}
  Let $r_i\ge2$, $i=1,\dots,n$, 
  and consider $f^{(i)}:(\R^d)^{r_i}\to\R$
  measurable functions that vanish on diagonals,
  i.e. $f^{(i)} (x_1,\dots,x_{r_i})=0$ whenever $x_k=x_l$ for some $1\le k\neq l\le r_i$, $i=1,\ldots , n$. We have   
\begin{equation}
  \label{fjkldf4}
  \E\left[\prod_{i=1}^n
     \int_{(\R^d)^{r_i}}f^{(i)} (z_1,\dots,z_{r_i})
  \ \eta(\mathrm{d}z_1)\cdots\eta(\mathrm{d}z_{r_i})  
\right]=
  \sum_{\substack{\rho\in\Pi ( \pi_1 \cup \cdots \cup \pi_n ) 
        \\\rho\wedge\pi=\widehat{0}} \atop {\rm (non-flat) \atop
  }}
\lambda^{|\rho|}\int_{(\R^d)^{|\rho|}}\left(\bigotimes_{i=1}^n f^{(i)} \right)_{\hskip-0.1cm \rho}(\mathbf{x})\mathrm{d}\mathbf{x}.
\end{equation}
\end{corollary}
% The main result of this section is a cumulant identity for multiple subgraph counts in the Poisson RCM.
 In Definition~\ref{fjmklc3},
 for every $\rho\in\Pi(\pi_1\cup \cdots \cup \pi_n )$
 we build a graph structure induced by
 $(G_1,\dots,G_n)$ on the diagram $\Gamma(\rho,\pi)$,
 analogous to \cite[Definition~2.3]{LiuPrivault}.
  % , and denote the diagram resulted in as $\rho_G$.
\begin{definition}
   \label{fjmklc3} 
 For $i=1,\dots,n$, let $G_i=(V_{G_i},E_{G_i})$
 be a connected graph with vertex set of the form 
 $V_{G_i}=\big(v^{(i)}_1, \ldots ,v^{(i)}_{r_i}; e_1,\ldots , e_m\big)$,
 and let $G:=\{G_1,\ldots , G_n\}$.
 Given $\rho \in\Pi(\pi_1\cup \cdots \cup \pi_n )$
 a partition of $\pi_1\cup \cdots \cup \pi_n $, 
 we let $\widetilde{\rho}_G$ denote the multigraph 
% from $n\times r + m$ nodes denoted respectively by $(i,j) \in [n] \times [r]$, and $(j) \in [m]$,
 constructed as follows on $[m] \cup \pi_1\cup \cdots \cup \pi_n$: 
\begin{enumerate}[i)]  
\item for all $j_1, j_2\in [r_i]$, $j_1\not= j_2$, and $i\in [n]$, 
  an edge links $(i,j_1)$ to $(i,j_2)$
  iff $\{v_{j_1},v_{j_2}\}\in E_{G_i}$. 
\item for all $(j,k)\in [r_i]\times [m]$ and $i\in [n]$, an edge
  links $(k)$ to $(i,j)$ iff $\{v_j,e_k\}\in E_{G_i}$; 
\item for all $i_1,i_2\in [n]$ and 
  $(j_1,j_2) \in [r_{i_1}]\times [r_{i_2}]$,
  we merge any two nodes $(i_1,j_1)$ and $(i_2,j_2)$ 
  if they belong to a same block in $\rho$. 
\end{enumerate}
In addition, we let $\rho_G$ be the graph constructed
from $\widetilde{\rho}_G$ on the blocks of $\rho$
by removing any redundant edge
in $\widetilde{\rho}_G$,
  so that at most one edge remains between any two blocks $\rho_1,\rho_2\in\rho$. 
\end{definition}
As in Section~\ref{diagramrepresentation},
 the graph $\rho_G$ forms a connected graph with
 $|\rho | + m$ vertices. 
 Figure~\ref{fig:diagram5} presents two examples of
 multigraphs $\widetilde{\rho}_G$ and graphs $\rho_G$ when $G_1,G_2,G_3$ are line graphs, $G_4$ is a triangle, and $G_5$ is a rectangle 
 on a partition diagram $\Gamma ( \rho , \pi )$ with
 no endpoints, i.e. $m=0$ here.
 
\begin{figure}[H]
\captionsetup[subfigure]{font=footnotesize}
\centering
\subcaptionbox{Multigraph $\widetilde{\rho}_G$ in blue.}[.5\textwidth]{%
\begin{tikzpicture}[hide labels,scale=0.9]
\tikzstyle{VertexStyle}=[shape = circle, fill = blue!20, minimum size = 0pt, scale=0., text = white, hide labels]
\draw[black, thick] (0,0) rectangle (5,6);
\node[anchor=east,font=\small] at (0.8,5) {1};
\node[anchor=east,font=\small] at (0.8,4) {2};
\node[anchor=east,font=\small] at (0.8,3) {3};
\node[anchor=east,font=\small] at (0.8,2) {4};
\node[anchor=east,font=\small] at (0.8,1) {5};
\node[anchor=south,font=\small] at (1,0) {1};
\node[anchor=south,font=\small] at (2,0) {2};
\node[anchor=south,font=\small] at (3,0) {3};
\node[anchor=south,font=\small] at (4,0) {4};
\filldraw [gray] (1,1) circle (2pt);
\filldraw [gray] (2,1) circle (2pt);
\filldraw [gray] (3,1) circle (2pt);
\filldraw [gray] (4,1) circle (2pt);
\filldraw [gray] (1,2) circle (2pt);
\filldraw [gray] (2,2) circle (2pt);
\filldraw [gray] (3,2) circle (2pt);
%\filldraw [gray] (4,2) circle (2pt);
\filldraw [gray] (1,3) circle (2pt);
\filldraw [gray] (2,3) circle (2pt);
\filldraw [gray] (3,3) circle (2pt);
\filldraw [gray] (4,3) circle (2pt);
\filldraw [gray] (2,3) circle (2pt);
\filldraw [gray] (1,4) circle (2pt);
\filldraw [gray] (2,4) circle (2pt);
%\filldraw [gray] (3,4) circle (2pt);
%\filldraw [gray] (4,4) circle (2pt);
\filldraw [gray] (1,5) circle (2pt);
\filldraw [gray] (2,5) circle (2pt);
\filldraw [gray] (3,5) circle (2pt);
%\filldraw [gray] (4,5) circle (2pt);
\draw[very thick] (1,5) -- (1,4);
\draw[very thick] (2,5) -- (2,4);
\draw[very thick] (2,5) -- (3,5);
\draw[very thick] (1,2) -- (1,1);
\draw[very thick] (2,3) -- (2,2);
\draw[very thick] (2,1) -- (3,1) -- (4,1);
\draw[very thick] (3,2) -- (4,3);
\node (1) [label=above:{}] at (1,5) {};
\node (2) [label=above:{}] at (2,5) {};
\node (3) [label=above:{}] at (3,5) {};
\node (4) [label=above:{}] at (4,5) {};
\node (5) [label=above:{}] at (1,4) {};
\node (6) [label=above:{}] at (2,4) {};
\node (7) [label=above:{}] at (3,4) {};
\node (8) [label=above:{}] at (4,4) {};
\draw[thick,dash dot,blue] (1,5) .. controls (1.5,5.5) .. (2,5);
\draw[thick,dash dot,blue] (1,4) .. controls (1.5,4.5) .. (2,4);
\draw[thick,dash dot,blue] (1,3) .. controls (1.5,3.5) .. (2,3);
\draw[thick,dash dot,blue] (2,3) .. controls (2.5,3.5) .. (3,3);
\draw[thick,dash dot,blue] (3,3) .. controls (3.5,3.5) .. (4,3);
\draw[thick,dash dot,blue] (1,2) .. controls (1.5,2.5) .. (2,2);
\draw[thick,dash dot,blue] (2,2) .. controls (2.5,2.5) .. (3,2);
\draw[thick,dash dot,blue] (1,2) .. controls (2,1.5) .. (3,2);
\draw[thick,dash dot,blue] (1,1) .. controls (1.5,1.5) .. (2,1);
\draw[thick,dash dot,blue] (4,1) .. controls (2.5,0.5) .. (1,1);

\end{tikzpicture}}%
\subcaptionbox{Graph $\rho_G$ in red.}[.5\textwidth]{
\begin{tikzpicture}[scale=0.9] 
\draw[black, thick] (0,0) rectangle (5,6);
\node[anchor=east,font=\small] at (0.8,5) {1};
\node[anchor=east,font=\small] at (0.8,4) {2};
\node[anchor=east,font=\small] at (0.8,3) {3};
\node[anchor=east,font=\small] at (0.8,2) {4};
\node[anchor=east,font=\small] at (0.8,1) {5};
\node[anchor=south,font=\small] at (1,0) {1};
\node[anchor=south,font=\small] at (2,0) {2};
\node[anchor=south,font=\small] at (3,0) {3};
\node[anchor=south,font=\small] at (4,0) {4};
\filldraw [gray] (1,1) circle (2pt);
\filldraw [gray] (2,1) circle (2pt);
\filldraw [gray] (3,1) circle (2pt);
\filldraw [gray] (4,1) circle (2pt);
\filldraw [gray] (1,2) circle (2pt);
\filldraw [gray] (2,2) circle (2pt);
\filldraw [gray] (3,2) circle (2pt);
%\filldraw [gray] (4,2) circle (2pt);
\filldraw [gray] (1,3) circle (2pt);
\filldraw [gray] (2,3) circle (2pt);
\filldraw [gray] (3,3) circle (2pt);
\filldraw [gray] (4,3) circle (2pt);
\filldraw [gray] (2,3) circle (2pt);
\filldraw [gray] (1,4) circle (2pt);
\filldraw [gray] (2,4) circle (2pt);
%\filldraw [gray] (3,4) circle (2pt);
%\filldraw [gray] (4,4) circle (2pt);
\filldraw [gray] (1,5) circle (2pt);
\filldraw [gray] (2,5) circle (2pt);
\filldraw [gray] (3,5) circle (2pt);
%\filldraw [gray] (4,5) circle (2pt);
\draw[very thick] (1,5) -- (1,4);
\draw[very thick] (2,5) -- (2,4);
\draw[very thick] (2,5) -- (3,5);
\draw[very thick] (1,2) -- (1,1);
\draw[very thick] (2,3) -- (2,2);
\draw[very thick] (2,1) -- (3,1) -- (4,1);
\draw[very thick] (3,2) -- (4,3);
\node (1) [label=above:{}] at (1,5) {};
\node (2) [label=above:{}] at (2,5) {};
\node (3) [label=above:{}] at (3,5) {};
\node (4) [label=above:{}] at (4,5) {};
\node (5) [label=above:{}] at (1,4) {};
\node (6) [label=above:{}] at (2,4) {};
\node (7) [label=above:{}] at (3,4) {};
\node (8) [label=above:{}] at (4,4) {};
\draw[thick,dash dot,red] (1,5) .. controls (1.5,5.5) .. (2,5);
%\draw[thick,dash dot,blue] (1,4) .. controls (1.5,4.5) .. (2,4);
\draw[thick,dash dot,red] (1,3) .. controls (1.5,3.5) .. (2,3);
\draw[thick,dash dot,red] (2,3) .. controls (2.5,3.5) .. (3,3);
\draw[thick,dash dot,red] (3,3) .. controls (3.5,3.5) .. (4,3);
\draw[thick,dash dot,red] (1,2) .. controls (1.5,2.5) .. (2,2);
\draw[thick,dash dot,red] (2,2) .. controls (2.5,2.5) .. (3,2);
\draw[thick,dash dot,red] (1,2) .. controls (2,1.5) .. (3,2);
\draw[thick,dash dot,red] (1,1) .. controls (1.5,1.5) .. (2,1);

\end{tikzpicture}}%
\vskip0.4cm
\captionsetup[subfigure]{font=footnotesize}
\centering
\subcaptionbox{Multigraph $\widetilde{\rho}_G$ in blue.}[.5\textwidth]{
\begin{tikzpicture}[scale=0.9] 
\draw[black, thick] (0,0) rectangle (5,6);

\node[anchor=east,font=\small] at (0.8,5) {1};
\node[anchor=east,font=\small] at (0.8,4) {2};
\node[anchor=east,font=\small] at (0.8,3) {3};
\node[anchor=east,font=\small] at (0.8,2) {4};
\node[anchor=east,font=\small] at (0.8,1) {5};

\node[anchor=south,font=\small] at (1,0) {1};
\node[anchor=south,font=\small] at (2,0) {2};
\node[anchor=south,font=\small] at (3,0) {3};
\node[anchor=south,font=\small] at (4,0) {4};

\filldraw [gray] (1,1) circle (2pt);
\filldraw [gray] (2,1) circle (2pt);
\filldraw [gray] (3,1) circle (2pt);
\filldraw [gray] (4,1) circle (2pt);
\filldraw [gray] (1,2) circle (2pt);
\filldraw [gray] (2,2) circle (2pt);
\filldraw [gray] (3,2) circle (2pt);
%\filldraw [gray] (4,2) circle (2pt);
\filldraw [gray] (1,3) circle (2pt);
\filldraw [gray] (2,3) circle (2pt);
\filldraw [gray] (3,3) circle (2pt);
\filldraw [gray] (4,3) circle (2pt);
\filldraw [gray] (2,3) circle (2pt);
\filldraw [gray] (1,4) circle (2pt);
\filldraw [gray] (2,4) circle (2pt);
%\filldraw [gray] (3,4) circle (2pt);
%\filldraw [gray] (4,4) circle (2pt);
\filldraw [gray] (1,5) circle (2pt);
\filldraw [gray] (2,5) circle (2pt);
\filldraw [gray] (3,5) circle (2pt);
%\filldraw [gray] (4,5) circle (2pt);

\draw[very thick] (1,5) -- (1,4); 
%\draw[very thick] (3,5) -- (4,4);

\draw[very thick] (1,2) -- (1,1);
\draw[very thick] (2,2) -- (2,4);
\draw[very thick] (2,1) -- (3,2) -- (4,3);
\draw[thick,dash dot,blue] (1,5) .. controls (1.5,5.5) .. (2,5);
\draw[thick,dash dot,blue] (2,5) .. controls (2.5,5.5) .. (3,5);
\draw[thick,dash dot,blue] (1,4) .. controls (1.5,4.5) .. (2,4);
\draw[thick,dash dot,blue] (1,3) .. controls (1.5,3.5) .. (2,3);
\draw[thick,dash dot,blue] (2,3) .. controls (2.5,3.5) .. (3,3);
\draw[thick,dash dot,blue] (3,3) .. controls (3.5,3.5) .. (4,3);
\draw[thick,dash dot,blue] (1,2) .. controls (1.5,2.5) .. (2,2);
\draw[thick,dash dot,blue] (2,2) .. controls (2.5,2.5) .. (3,2);
\draw[thick,dash dot,blue] (1,2) .. controls (2,1.5) .. (3,2);
\draw[thick,dash dot,blue] (1,1) .. controls (1.5,1.5) .. (2,1);
\draw[thick,dash dot,blue] (2,1) .. controls (2.5,1.5) .. (3,1);
\draw[thick,dash dot,blue] (3,1) .. controls (3.5,1.5) .. (4,1);
\draw[thick,dash dot,blue] (1,1) .. controls (2.5,0.5) .. (4,1);

\end{tikzpicture}}%
\subcaptionbox{Graph and $\rho_G$ in red.}[.5\textwidth]{
\begin{tikzpicture}[scale=0.9] 
\draw[black, thick] (0,0) rectangle (5,6);

\node[anchor=east,font=\small] at (0.8,5) {1};
\node[anchor=east,font=\small] at (0.8,4) {2};
\node[anchor=east,font=\small] at (0.8,3) {3};
\node[anchor=east,font=\small] at (0.8,2) {4};
\node[anchor=east,font=\small] at (0.8,1) {5};

\node[anchor=south,font=\small] at (1,0) {1};
\node[anchor=south,font=\small] at (2,0) {2};
\node[anchor=south,font=\small] at (3,0) {3};
\node[anchor=south,font=\small] at (4,0) {4};

\filldraw [gray] (1,1) circle (2pt);
\filldraw [gray] (2,1) circle (2pt);
\filldraw [gray] (3,1) circle (2pt);
\filldraw [gray] (4,1) circle (2pt);
\filldraw [gray] (1,2) circle (2pt);
\filldraw [gray] (2,2) circle (2pt);
\filldraw [gray] (3,2) circle (2pt);
%\filldraw [gray] (4,2) circle (2pt);
\filldraw [gray] (1,3) circle (2pt);
\filldraw [gray] (2,3) circle (2pt);
\filldraw [gray] (3,3) circle (2pt);
\filldraw [gray] (4,3) circle (2pt);
\filldraw [gray] (2,3) circle (2pt);
\filldraw [gray] (1,4) circle (2pt);
\filldraw [gray] (2,4) circle (2pt);
%\filldraw [gray] (3,4) circle (2pt);
%\filldraw [gray] (4,4) circle (2pt);
\filldraw [gray] (1,5) circle (2pt);
\filldraw [gray] (2,5) circle (2pt);
\filldraw [gray] (3,5) circle (2pt);
%\filldraw [gray] (4,5) circle (2pt);

\draw[very thick] (1,5) -- (1,4); 
%\draw[very thick] (3,5) -- (4,4);

\draw[very thick] (1,2) -- (1,1);
\draw[very thick] (2,2) -- (2,4);
\draw[very thick] (2,1) -- (3,2) -- (4,3);
\draw[thick,dash dot,red] (1,5) .. controls (1.5,5.5) .. (2,5);
\draw[thick,dash dot,red] (2,5) .. controls (2.5,5.5) .. (3,5);
\draw[thick,dash dot,red] (1,4) .. controls (1.5,4.5) .. (2,4);
\draw[thick,dash dot,red] (1,3) .. controls (1.5,3.5) .. (2,3);
\draw[thick,dash dot,red] (2,3) .. controls (2.5,3.5) .. (3,3);
\draw[thick,dash dot,red] (3,3) .. controls (3.5,3.5) .. (4,3);
\draw[thick,dash dot,red] (1,2) .. controls (1.5,2.5) .. (2,2);
\draw[thick,dash dot,red] (2,2) .. controls (2.5,2.5) .. (3,2);
\draw[thick,dash dot,red] (1,2) .. controls (2,1.5) .. (3,2);
%\draw[thick,dash dot,red] (1,1) .. controls (1.5,1.5) .. (2,1);
\draw[thick,dash dot,red] (2,1) .. controls (2.5,1.5) .. (3,1);
\draw[thick,dash dot,red] (3,1) .. controls (3.5,1.5) .. (4,1);
\draw[thick,dash dot,red] (1,1) .. controls (2.5,0.5) .. (4,1);

\end{tikzpicture}}%
\caption{Diagram $\Gamma ( \rho , \pi )$,
  multigraph $\widetilde{\rho}_G$
  and graph $\rho_G$.}
\label{fig:diagram5}
\end{figure}
\vspace{-0.4cm}
\noindent
Denote by $N_{y_1,\ldots , y_m}^{G_i}$ the count of subgraphs in
the random-connection model $G_H (\eta \cup \{y_1,\ldots , y_m \} )$,
i.e.
\begin{equation}
\nonumber
N_{y_1,\ldots , y_m}^{G_i}
=\sum_{(x_1, \ldots ,x_{r_i})\in\eta^{r_i}} f^{(i)}_{y_1,\ldots , y_m} (x_1, \ldots ,x_{r_i}), 
\end{equation}
 where
 $f^{(i)}_{y_1,\ldots , y_m} :(\real^d)^{r_i} \to \{0,1\}$ is the random function defined as 
\begin{equation}
\nonumber
f^{(i)}_{y_1,\ldots , y_m} (x_1, \ldots ,x_{r_i}):=
\prod_{  \substack{
    1 \leq \ell \leq r_i
    \\
    1 \leq j \leq m
    \\ \{v_\ell,e_j\}\in E_{G_i} }
}
\bone_{\{y_j\leftrightarrow x_i\}} 
\prod_{\substack{ 1 \leq k,l \leq r_i
    \\ \{v_\ell,v_k\}\in E_{G_i}}}\bone_{\{x_\ell\leftrightarrow x_k\}},
\qquad
 x_1,\ldots , x_{r_i} \in \R^d. 
\end{equation} 
% Recall definitions from Section~\ref{diagramrepresentation}, and let $G_i=(V_{G_i},E_{G_i})$ be a connected graph with edge set $E_{G_i}$ and vertex set $V_{G_i}=\{v_1,\dots,v_{r_i},e_1,\dots,e_m\}$, for $i=1,\dots,n$. For $i=1,\dots,n$, we define
% The next result is a generalisation of Proposition~\ref{mom-cumfor}. 
 For $\rho = \{ b_1,\ldots , b_{|\rho |}\}
\in\Pi ([n]\times[r])$, we also let 
\begin{equation}
\nonumber
    {\cal A}^\rho_j:=\{ k \in [ |\rho | ] \ : \ \exists (s,i)\in b_k ~\mathrm{s.t.}~
    (v_i,e_j) \in E_{G_i} % , \ s\in [n], \ i\in [r]
    \} 
\end{equation} 
denote the neighborhood of the vertex $(|\rho | + j)$ in $\rho_G$,
$j=1,\ldots , m$.
\begin{prop}
  \label{fjklf2}
  Let $  N_{y_1,\ldots , y_m}^{G_i}$ be subgraph counts in the
  random-connection model $G_H (\eta\cup\{y_1,\dots,y_m\})$ as defined above, for $i=1,\dots,n$. We have 
    \begin{equation}\label{moment-4}
      \E\left[\prod_{i=1}^nN^{G_i}_{y_1\dots,y_m}\right]=\sum_{
        \substack{
          \rho\in\Pi(\pi_1 \cup \cdots \cup \pi_n )
      \\\rho\wedge\pi=\widehat{0}} \atop {\rm (non-flat)}}
\lambda^{|\rho|}\int_{(\R^d)^{|\rho|}}\prod_{\substack{ % 1 \leq i \leq |\rho| \\
      1 \leq j \leq m
      \\ i\in {\cal A}^\rho_j}}
    H(x_i,y_j)\prod_{(k,l)\in E_{\rho_G} }H(x_k,x_l)\mu(\mathrm{d}\mathbf{x}),
\end{equation}
and joint cumulant 
\begin{equation}
  \label{cum-2}
  \kappa(N_{G_1},\dots,N_{G_n})=\sum_{
    \substack{\rho\in\Pi_{\widehat{1}}(\pi_1 \cup \cdots \cup \pi_n )
        \\\rho\wedge\pi=\widehat{0}} \atop {\rm (non-flat \ \! connected)}}
  \lambda^{|\rho|}\int_{(\R^d)^{|\rho|}}
  \prod_{\substack{ % 1 \leq i \leq |\rho| \\
      1 \leq j \leq m
      \\ i\in {\cal A}^\rho_j}}
    H(x_i,y_j)\prod_{(k,l)\in E_{\rho_G} }H(x_k,x_l)\mu(\mathrm{d}\mathbf{x}). 
\end{equation}
\end{prop}
\begin{Proof}
 The moment identity \eqref{moment-4} is obtained by
 taking expectation on both sides of
 \eqref{fjkldf4} in Corollary~\ref{moment-2}
 and using the relation $H(x,y) = \E [ 
 \bone_{\{x \leftrightarrow y\}}]$,
 $x, y \in \real^d$.
 Next, we note that 
 the connectedness factorization property
 \eqref{factor-1} is satisfied by
 $$F(\rho ):=\lambda^{|\rho |}\int_{(\R^d)^{|\rho |}}
 \prod_{\substack{ % 1 \leq i \leq |\rho| \\
      1 \leq j \leq m
      \\ i\in {\cal A}^\rho_j}}
 H(x_i,y_j)
 \prod_{(k,l)\in E_{\rho_G} }H(x_k,x_l)\mu(\mathrm{d}x_1)\cdots\mu(\mathrm{d}x_{|\rho |}),
 $$
 $\rho \in\Pi(\pi_1\cup \cdots \cup \pi_n )$,
 hence \eqref{cum-2} follows from
 Relations~\eqref{mm}, \eqref{moment-4},
 \eqref{eq-1}, 
 and the classical cumulant-moment relationship \eqref{2},
 see e.g. Relation~(3.3) in \cite{elukacs}. 
\end{Proof}
\noindent
The cumulant formula of Proposition~\ref{fjklf2}
is implemented in the code given in Appendix~\ref{fjkldsf-2}. 
\section{Cumulant and factorial moment estimates} 
\label{statuleviciuscond}
\noindent
The following result can be found in
 \cite[Corollary~2.1]{saulis} or \cite[Theorem~2.4]{doering}. 
% We should mention that the assumption \eqref{Statuleviciuscond2} is now known as the {\it Statulevi\v{c}ius condition}.
\begin{lemma}\label{Statuleviciuscond1}
  Let $\{X_\lambda\}$ be a family of random variables with moments
  of all orders, mean zero and unit variance for all $\lambda>0$.
  Suppose that for all $j\geq 3$ and sufficiently large $\lambda$,
  the cumulant of order $j$ of $X_\lambda$ is bounded by
\begin{equation}
\nonumber %   \label{Statuleviciuscond2}
|\kappa_j(X_\lambda)|\leq \frac{(j!)^{1+\gamma}}{(\Delta_\lambda)^{j-2}}
\end{equation}
where $\gamma\ge0$ is a constant independent of $\lambda$.
Then we have the Berry-Esseen bound 
\begin{equation}
\nonumber
  \sup_{x\in\R}|\IP (X_\lambda\leq x)-\Phi(x)|\leq C_\gamma (\Delta_\lambda)^{-1/(1+2\gamma)},
\end{equation}
 for $C_\gamma>0$ a constant depending only on $\gamma$. 
\end{lemma}
Let
$m_n(X) := \E [ X(X-1) \cdots (X-n+1)]$,  
% = \sum_{k=1}^n s(n,k) \E [ X^k],
 $n \geq 1$, 
% $m_n(X)$ of order $n$ 
 denote the factorial moments
 of a discrete random variable $X$. 
 % , which are defined by 
% where $s(n,k)$ are the Stirling numbers of the first kind. 
\begin{prop}[Corollary 1.13 in \cite{bollobas}] 
  \label{fdshkf0}
  Assume that 
  $$
  \lim_{n\to \infty} m_n(X)\frac{n^m}{n!} =0,
  \qquad
 m\geq 0.
  $$ 
 Then for any $n\geq 0$, we have 
 \begin{equation}
   \label{fjkl32}
   \P (   X = n ) = 
\frac{1}{n!}
\sum_{i\geq 0} \frac{(-1)^i}{i!}
m_{n + i}(X).
\end{equation}
\end{prop}
\section{Gram-Charlier expansions}
\label{s5}
\noindent
 Let $\varphi(x) := e^{-x^2/2} / \sqrt{2\pi}$,
 $x\in \real$,
 denote the standard normal probability density function. 
 In addition to the second order expansion Gaussian approximation 
\begin{equation} 
\label{gram_charlier-1}
\phi_X^{(1)}(x)=
\frac{1}{\sqrt{\kappa_2}}
\varphi \left( \frac{x-\kappa_1}{\sqrt{\kappa_2}}\right)
\end{equation}
for the probability density $\phi_X(x)$
function of a random variable $X$,
higher order Gram-Charlier expansions of 
 third and fourth order are given by 
\begin{equation} 
\label{gram_charlier-2}
\phi_X^{(3)}(x)=
\frac{1}{\sqrt{\kappa_2}}
\varphi \left( \frac{x-\kappa_1}{\sqrt{\kappa_2}}\right)
\left( 1 +
c_3 H_3\left(
\frac{x-\kappa_1}{\sqrt{\kappa_2}} \right)
\right)
\end{equation}
 and
\begin{equation} 
\nonumber % \\label{fourth} 
\phi_X^{(4)}(x)=
\frac{1}{\sqrt{\kappa_2}}
\varphi \left( \frac{x-\kappa_1}{\sqrt{\kappa_2}}\right)
\left( 1 +
c_3 H_3\left(
\frac{x-\kappa_1}{\sqrt{\kappa_2}} \right)
+
c_4 H_4\left(
\frac{x-\kappa_1}{\sqrt{\kappa_2}} \right)
 + c_6 H_6\left( \frac{x-\kappa_1}{\sqrt{\kappa_2}} \right)
 \right).
\end{equation}
 see \S~17.6 of \cite{cramer}, where 
\begin{itemize}
\item 
$H_0(x)=1$, 
$H_1(x)=x$, 
$H_3(x)=x^3-3x$,
$H_4(x)=x^4-6x^2+3$, 
$H_6(x)=x^6-15x^4+45x^2-15$ 
 are Hermite polynomials, 
 \item 
  the sequence $c_3,c_4,c_5,c_6$ is given from the cumulants $(\kappa_n)_{n\geq 1}$
of $X$ as 
$$ 
c_3 = \frac{\kappa_3}{3! (\kappa_2)^{3/2}}, 
\quad
c_4 = \frac{\kappa_4}{4! (\kappa_2)^2}, 
\quad 
c_5 = \frac{\kappa_5}{5! \kappa_5^{5/2}},
\quad
 c_6 =
 \frac{\kappa_6}{6! (\kappa_2)^3}
 +
  \frac{(\kappa_3)^2}{2(3!)^2 (\kappa_2)^3}, 
$$
 where $c_3$ and $c_4$ are expressed from 
 the skewness $\kappa_3/(\kappa_2)^{3/2}$ and
 the excess kurtosis $\kappa_4/(\kappa_2)^2$.
\end{itemize} 
% \noindent
\noindent

\section{Cumulant code}
\label{fjkldsf}
\noindent  
The following code generates closed-form cumulant expressions 
via symbolic calculations in SageMath for any dimension $d\geq 1$,
any connected subgraph $G$ and any set of endpoints $\EE$. 
The choice of SageMath for this implementation
is due to its fast handling of symbolic integration via Maxima, which is significantly faster than the Python package Sympy. 
This code is also sped up by parallel processing that
distributes the load among multiple CPU cores. 
% See https://news.ycombinator.com/item?id=23767233
This SageMath code and the next one
are available for download at
\url{https://github.com/nprivaul/random-connection}.  
This code also covers the case where $G$ has no endpoints ($m=0$)
by taking $\EE :=[ \ ]$ empty. However,
in this case $\mu$ should be a finite measure, i.e., 
the density function mu$(x,\lambda ,\beta )$ should be
integrable with respect to the Lebesgue measure.

\bigskip

\begin{lstlisting}
from time import time
import datetime
import multiprocessing as mp

global cumulants

def partitions(points):
    if len(points) == 1:
        yield [ points ]
        return
    first = points[0]
    for smaller in partitions(points[1:]):
        for m, subset in enumerate(smaller):
            yield smaller[:m] + [[ first ] + subset]  + smaller[m+1:]
        yield [ [ first ] ] + smaller

def nonflat(partition,r):
    p = []
    for j in partition:    
        seq = list(map(lambda x: (x-1)//r,j))
        p.append(len(seq) == len(set(seq)))
    return all(p)

def connected(partition,n,r):
    q = []; c = 0
    if n  == 1: return all([len(j)==1 for j in partition])
    for j in partition:
        jk = list(set(map(lambda x: (x-1)//r,j)))
        if(len(jk)>1):            
            if c == 0:
                q = jk; c += 1
            elif(set(q) & set(jk)):
                d=[y for y in (q+jk) if y not in q]
                q = q + d
    return n == len(set(q))

def connectednonflat(n,r):
    points = list(range(1,n*r+1))
    randd = []
    for m, p in enumerate(partitions(points), 1):
        randd.append(sorted(p))
    cnfp = [e for e in randd if (connected(e,n,r) and nonflat(e,r))]
    for rou in range(r,(r-1)*n+2): 
        rs = [d for d in cnfp if len(d)==rou]
        print("Connected non-flat partitions with",rou,"blocks:",len(rs))
    print("Connected non-flat set partitions:",len(cnfp))
    return cnfp

def graphs(G,E,setpartition,n):
    r=len(set(flatten(G)));rhoG = []
    for j in range(n):
        for hop in G: rhoG.append([r*j+hop[0],r*j+hop[1]])
        for l in range(len(E)):
            F=E[l]
            for i in F: rhoG.append([j*r+i,n*r+l+1]);
    for i in setpartition:
        if(len(i)>1):
            b = []
            for j in rhoG:
                b.append([i[0] if ele in i else ele for ele in j])
            rhoG = b
    for i in rhoG: i.sort()
    return rhoG

def inner(n,d,G,E,mu,H,setpartition,z,r):
    rhoG=graphs(G,E,setpartition,n)
    for ll in range(len(E)+1):
        for l in range(1,d+1): z[d*(n*r+ll)+l] = var(str(y)+str(ll)+str('_')+str(l))
    for key in range(1,n*r+1): 
        for l in range(1,d+1): z[key*d+l] = var(str(x)+str(key)+str(x)+str(l))
    edgesrhoG = [i for n, i in enumerate(rhoG) if i not in rhoG[:n]]
    vertrhoG = set(flatten(edgesrhoG));
    for ll in range(len(E)): vertrhoG.remove(n*r+ll+1);
    strr = '*λ'*len(vertrhoG)
    for i in vertrhoG:
        for l in range(1,d+1): strr = '*mu({},{},{})'.format(z[i*d+l],λ,β) + strr
        for l in range(1,d+1): strr = strr + ').integrate({},-infinity,+infinity)'.format(z[i*d+l])
    for i in edgesrhoG:
        for l in range(1,d+1): strr = '*H({},{},{})'.format(z[i[0]*d+l],z[i[1]*d+l],β) + strr
    strr = '('*len(vertrhoG)*d+strr[1:]
    return eval(preparse(strr))

def collect_result(result):
    global cumulants
    global iii
    global tim
    iii=iii+1;
    if (mod(iii,100)==0):
        tim=(time()-t_start2)*(lencnfp-iii)/iii/60
        print('[%d]\r'%(iii),'Est remaining time (minutes):%d'%(tim),end="")
    cumulants+=result
	
def c(n,d,G,E,mu,H):
    global cumulants
    global iii
    global t_start2
    t_start2 = time()
    d_start2 = datetime.datetime.now()
    r=len(set(flatten(G)));
    x,y=var("x,y")
    cumulants = 0; iii = 0
    z = dict(enumerate([str(x)+str(key)+str(x)+str(l) for key in range(0,n*r+1) for l in range(1,d+1)], start=1))
    global lencnfp
    cnfp=connectednonflat(n,r)
    lencnfp=len(cnfp)
    pool = mp.Pool(4) # pool = mp.Pool(mp.cpu_count())
    for setpartition in cnfp: 
        pool.apply_async(func = inner, args=(n,d,G,E,mu,H,setpartition,z,r), callback=collect_result)
    pool.close()
    pool.join()
    print("\n");
    d_end2 = datetime.datetime.now()
    print("Runtime is",(d_end2-d_start2))
    return cumulants._sympy_() 
\end{lstlisting}

\section{Joint cumulant code}
\label{fjkldsf-2}
\noindent  
The following code generates closed-form joint cumulant expressions 
via symbolic calculations in SageMath for any dimension $d\geq 1$,
any sequence $(G_1,\ldots ,G_n)$ of
connected subgraphs, and any sequence $(e_1,\ldots, e_m)$ of endpoints. 

% See https://news.ycombinator.com/item?id=23767233

\medskip

\begin{lstlisting}
def jpartitions(points):
    if len(points) == 1:
        yield [ points ]
        return
    first = points[0]
    for smaller in jpartitions(points[1:]):
        for m, subset in enumerate(smaller):
            yield smaller[:m] + [[ first ] + subset]  + smaller[m+1:]
        yield [ [ first ] ] + smaller

def jnonflat(partition,rr):
    n=len(rr); p = []
    for j in partition:    
        for i in range(n):
            j2 = [l for l in j if l > sum(rr[0:i]) and l<=sum(rr[0:(i+1)])]
            p.append(len(j2) <= 1)
    return all(p)

def jconnected(partition,rr):
    n=len(rr); q = []; c = 0; 
    if n  == 1: return True
    for j in partition:
        jk = [i for i in range(n) if len([l for l in j if l > sum(rr[0:i]) and l<=sum(rr[0:(i+1)])])>=1]
        if(len(jk)>1):            
            if c == 0:
                q = jk; c += 1
            elif(set(q) & set(jk)):
                d=[y for y in (q+jk) if y not in q]
                q = q + d
    return n == len(q)

def jconnectednonflat(rr):
    n=len(rr); 
    points = list(range(1,sum(rr)+1))
    randd = []
    for m, p in enumerate(jpartitions(points), 1): randd.append(sorted(p))
    for rou in range(min(rr),sum(rr)-n+2):    
        rs = [d for d in randd if (jnonflat(d,rr) and len(d)==rou)]
        rss = [e for e in rs if jconnected(e,rr)]
        print("Connected non-flat partitions with",rou,"blocks:",len(rss))
    cnfp = [e for e in randd if (jconnected(e,rr) and jnonflat(e,rr))]
    print("Connected non-flat set partitions:",len(cnfp))
    return cnfp

def jgraphs(G,E,setpartition):
    rr=[len(set(flatten(g))) for g in G];
    n=len(G); rhoG = []
    ee=[len(set(flatten(e))) for e in E];
    for j in range(n):
        for hop in G[j]: rhoG.append([hop[0],hop[1]])
        for l in range(len(E)):
            F=E[l]
            for i in F: rhoG.append([i,sum(rr)+l+1]);
    for i in setpartition:
        if(len(i)>1):
            b = []
            for j in rhoG:
                b.append([i[0] if ele in i else ele for ele in j])
            rhoG = b
    for i in rhoG: i.sort()
    return rhoG

def jc(d,G,E,mu,H):
    rr=[len(set(flatten(g))) for g in G];
    if(sum(rr)!=len(set(flatten(G)))):
        print("Wrong G format");
        return 0
    n=len(G);
    ee=[len(set(flatten(e))) for e in E];
    x,y=var("x,y")
    jcumulants = 0; ii=0
    z = dict(enumerate([str(x)+str(key)+str(x)+str(l) for key in range(1,sum(rr)+1) for l in range(1,d+1)], start=1))
    cnfp=jconnectednonflat(rr)
    for setpartition in cnfp: 
        ii=ii+1;print('[%d/%d]\r'%(ii,len(cnfp)),end="")
        rhoG=jgraphs(G,E,setpartition)
        for j in range(n):
            m=len(E); 
            for l in range(m+1):
                for ld in range(1,d+1): z[d*(sum(rr)+l)+ld] = var(str(y)+str(l)+str('_')+str(ld))
        for key in range(1,sum(rr)+1): 
            for l in range(1,d+1): z[key*d+l] = var(str(x)+str(key)+str(x)+str(l))
        edgesrhoG = [i for n, i in enumerate(rhoG) if i not in rhoG[:n]]
        vertrhoG = set(flatten(edgesrhoG));
        m=len(E); 
        for l in range(m): vertrhoG.remove(sum(rr)+l+1);
        strr = '*λ'*len(vertrhoG)
        for i in vertrhoG:
            for l in range(1,d+1): strr = '*mu({},{},{})'.format(z[i*d+l],λ,β) + strr
            for l in range(1,d+1): strr = strr + ').integrate({},-infinity,+infinity)'.format(z[i*d+l])
        for i in edgesrhoG:
            for l in range(1,d+1): strr = '*H({},{},{})'.format(z[i[0]*d+l],z[i[1]*d+l],β) + strr
        strr = '('*len(vertrhoG)*d+strr[1:]
        jcumulants += eval(preparse(strr))
    print("\n");
    jcumulants = simplify(jcumulants).canonicalize_radical().maxima_methods().rootscontract().simplify()
    return jcumulants._sympy_()
\end{lstlisting} 

% \subsubsection*{Availability of data and materials} 
% \noindent
%  No new data were created during the study.

% \subsubsection*{Conflicts of interest statement} 
% \noindent
%  The authors have no conflict of interest to declare. 

\subsubsection*{Acknowledgement}
\noindent
We thank Xueying Yang for essential contributions to the
SageMath cumulant codes. 

\footnotesize

\bibliography{../../bib/privault}
\bibliographystyle{alpha}
% \bibliographystyle{plainnat}

\end{document}

 In what follows, given $f:(\real^d)^r \to\R$ a measurable
 function, we consider the multivariate Poisson stochastic
 integral
 \begin{equation}
   \label{ps}
   \sum_{(x_1, \ldots ,x_r)\in\eta_r^{\ne}}f(x_1, \ldots ,x_r)
   , 
\end{equation}
 where 
 $$
 \eta_r^{\ne}:=\{(x_1, \ldots ,x_r) \ : \ x_i\in\eta
 \mbox{ and } x_i\ne x_j~\mathrm{for}~ 1 \leq i\ne j \leq r\} 
 $$
 denotes the collection of $r$-tuples in $\eta$ with $r$ distinct entries,
 $r\geq 1$. 
 Proposition~\ref{fjl} is a consequence of e.g. \cite[Proposition~2]{prkhp}.
 \begin{prop} 
  \label{fjl}
  Given $f:(\real^d)^r \to\R$ a sufficiently integrable
  function, we have 
\begin{equation} 
\label{fjkld2}
\E_\lambda \left[\left(
  \sum_{(x_1, \ldots ,x_r)\in\eta_r^{\ne}}f(x_1, \ldots ,x_r)
  \right)^n\right]=
\sum_{\substack{\rho\in\Pi ([n]\times[r])
    \\\rho\wedge\pi=\widehat{0}} \atop {\rm (non-flat)}}
\lambda^{|\rho|}
\int_{(\R^d)^{|\rho|}}
    \prod_{i=1}^n f(x_{i,1}, \ldots ,x_{i,r} ) \ \! 
    \mu^{|\rho|} ( \mathrm{d}\mathbf{x} ), 
\end{equation} 
% $$ \eta^r (\mathrm{d}\mathbf{x}):=\eta(\mathrm{d}x_1)\cdots\eta(\mathrm{d}x_r), 
 where for any $\rho \in\Pi ([n]\times[r])$ and
 $(i,j) \in [n] \times [r]$ we let 
 $\mathbf{x}:=(x_1 , \ldots ,x_{|\rho|})$ and 
 $$
 x_{i,j}:=x_\ell \mbox{ ~if~ } (i,j)\in\rho_\ell
 \mbox{ ~for some~ } \ell = 1, \ldots , |\rho |. 
$$
 % V_\rho (\mathbf{y},\eta_{y_1, \ldots ,y_{|\rho|}})
\end{prop}
 The next proposition is proved similarly to
 Proposition~4.3 in \cite{LiuPrivault} 
 as an application of Proposition~3.3 therein, 
 see also Lemma~2 in \cite[p.~34]{MalyshevMinlos91}
 and Lemma~3.1 in \cite{khorunzhiy}. 
 \begin{prop}
   \label{connectedcumulant-1}
 Given $f:(\real^d)^r \to\R$ a sufficiently integrable
 function, the cumulant of order $n\geq 1$ of \eqref{ps}
 is given by 
\begin{equation}
\label{connectedcumulant}
\kappa_n
\left(
  \sum_{(x_1, \ldots ,x_r)\in\eta_r^{\ne}}f(x_1, \ldots ,x_r)
  \right)
= 
\sum_{\substack{\rho\in\Pi_{\widehat{1}} ([n]\times[r])
    \\\rho\wedge\pi=\widehat{0}} \atop {\rm (non-flat \ \! connected)}}
   \lambda^{|\rho |}
        \int_{(\R^d)^{|\rho |}}
     \prod_{i=1}^nf(x_{i,1}, \ldots ,x_{i,r})
     \mu^{ |\rho |} ( \mathrm{d}\mathbf{x} ) 
   . 
\end{equation}
\end{prop}
\begin{Proof}
 Let $F$ be the function defined on $\Pi ([n]\times[r])$ as 
\begin{equation}
  \nonumber
  F(\rho):=\lambda^{|\rho|}\int_{(\R^d)^{|\rho|}}
     \prod_{i=1}^nf(x_{i,1}, \ldots ,x_{i,r})
     \mu^{ |\rho |} ( \mathrm{d}\mathbf{x} ) 
\end{equation}
 for $\rho=\{\rho_1,\dots,\rho_{|\rho|}\}$, 
 with $x_{i,j}:=x_\ell$ if $(i,j)\in\rho_\ell$,
 $(i,j) \in [n] \times [r]$, $\ell =1,\ldots , |\rho|$. 
 To show \eqref{connectedcumulant} it suffices to note that
 $F$ satisfies the connectedness factorization property in \cite[Definition~3.1]{LiuPrivault}, and the conclusion then follows from Proposition~\ref{fjl} above
 and Proposition~3.3 in \cite{LiuPrivault}. 
% \begin{equation}  \label{connectedcumulant}  \kappa_n\big(N_{y_1,\ldots , y_m}^G\big)  =\sum_{\substack{\sigma\in\Pi_{\widehat{1}}[n\times r]\\\sigma\wedge\pi=\widehat{0}}}F(\sigma),\end{equation}
\end{Proof} 
% \section{The case of two-hop paths} 





\subsubsection*{Two-hop paths with a single endpoint}
\noindent
 Here, we take $m=1$, $r=2$, and in Table~\ref{t1} we 
 compute the first three cumulants of $N^G_{y_1}$ when $G$ is
 a single-edge graph with a single endpoint in dimension $d=1$.  
% after loading the relevant function definitions. 

\begin{figure}[H]
  \centering
    \includegraphics[width=0.8\linewidth]{../end_points/end_points_multidim/one_endpoint_two_hops/plot} 
\caption{Two-hop paths with a single endpoint in dimension $d=2$.} 
\end{figure}

\vspace{-0.3cm} 

\begin{table}[H] 
  \centering
\scriptsize %   \small
%  \resizebox{\textwidth}{!}
    {
  \begin{tabular}{|ll|ll|} % {\textwidth}{|XX|XX|}
 \hline
 \multicolumn{2}{|l}{
G = [[1,2]];  E=[[1]]; d=1 
 }
 & \multicolumn{2}{l|}{\# Single edge graph $r=2$,
  single endpoint $m=1$, dimension $d=1$~~~~~~~~~~~~~~~~~~~~~~~~~~~~~~~~~~~~}   
 \\
\hline
\end{tabular}
}
  \resizebox{\textwidth}{!}{
  \begin{tabular}{|ll|ll|} % {\textwidth}{|XX|XX|}
\hline
\multicolumn{1}{|c|}{Instruction} & \multicolumn{1}{c|}{Order} & \multicolumn{1}{c|}{Cumulant output} & \multicolumn{1}{c|}{Connected non-flat partitions} 
 \\ 
 \hline
\multicolumn{1}{|c|}{c(1,d,G,E,mu,H)} & \multicolumn{1}{c|}{1st} & \multicolumn{1}{c|}{\large $\frac{\pi {\lambda}^{2}}{{\beta}}$} & \multicolumn{1}{c|}{1} 
\\
\hline
\multicolumn{1}{|c|}{c(2,d,G,E,mu,H)} & \multicolumn{1}{c|}{2nd} & \multicolumn{1}{c|}{\large $\frac{\pi^{{3}/{2}} {\lambda}^{3} {\left(4 \, \sqrt{3} + 9\right)} + 2 \, \pi \sqrt{{\beta}} {\lambda}^{2} {\left(\sqrt{3} + 3\right)}}{6 \, {\beta}^{{3}/{2}}}$} & \multicolumn{1}{c|}{6} 
\\
\hline
\multicolumn{1}{|c|}{c(3,d,G,E,mu,H)} & \multicolumn{1}{c|}{3rd} & \multicolumn{1}{c|}{
\large $\frac{\sqrt{22} {\left(\pi^{2} {\left(\sqrt{11} {\left(19 \, \sqrt{6} + 24 \, \sqrt{2} + 27\right)} + 36 \, \sqrt{2}\right)} {\lambda}^{4} + 6 \, \sqrt{11} \pi {\beta} {\lambda}^{2} {\left(\sqrt{6} + \sqrt{2}\right)} + 6 \, \sqrt{11} \sqrt{\pi^{3} {\beta}} {\lambda}^{3} {\left(4 \, \sqrt{6} + 5 \, \sqrt{2} + 6\right)}\right)}}{132 \, {\beta}^{2}}
$} & \multicolumn{1}{c|}{68} 
 \\ % [1ex]
\hline
\end{tabular}
}
\caption{Cumulants of the count of $2$-hop paths with one endpoint in dimension $d=2$.}
\label{t1}
\end{table} 

\vspace{-0.6cm}

\noindent
 In Figures~\ref{fig1-11}-\ref{fig2-11} we plot the corresponding
 moment expressions {\em vs} their Monte Carlo estimates
 in dimension $d=1$ with $y_1 = 0$ and $y_2 = 1$. 

\begin{figure}[H]
  \centering
 \begin{subfigure}[b]{0.49\textwidth}
    \includegraphics[width=1\linewidth, height=5cm]{../end_points/end_points_multidim/one_endpoint_two_hops/m1} 
    \caption{First moment.} 
 \end{subfigure}
 \begin{subfigure}[b]{0.49\textwidth}
    \includegraphics[width=1\linewidth, height=5cm]{../end_points/end_points_multidim/one_endpoint_two_hops/m2} 
    \caption{Second moment.} 
 \end{subfigure}
  \caption{First and second moments.} 
\label{fig1-11} 
\end{figure}
\vskip-0.4cm 
\begin{figure}[H]
  \centering
 \begin{subfigure}[b]{0.49\textwidth}
    \includegraphics[width=1\linewidth, height=5cm]{../end_points/end_points_multidim/one_endpoint_two_hops/m3} 
    \caption{Third moment.} 
 \end{subfigure}
 \begin{subfigure}[b]{0.49\textwidth}
    \includegraphics[width=1\linewidth, height=5cm]{../end_points/end_points_multidim/one_endpoint_two_hops/m4} 
    \caption{Fourth moment.} 
 \end{subfigure}
  \caption{Third and fourth moments.} 
\label{fig2-11} 
\end{figure}

Let $y_1, \ldots ,y_k\in \R^d$ be distinct.
We represent the number of
subgraphs isomorphic to a $k$-cycle
connecting with $k$ endpoints 
in the random-connection model $G_H (\eta_{y_1, \ldots ,y_k})$
as 
\begin{equation}
\nonumber
  N_3:=\sum_{(x_1, \ldots , x_k)\in\eta^k}\prod_{i=1}^k\left(\bone_{\{x_i\leftrightarrow x_{i+1}\}}\bone_{\{x_i\leftrightarrow y_i\}}\right),
\end{equation}
 where we set $x_{k+1}:=x_1$. 
% Set $\widetilde{N}_i$, $i=1,2,3$ as the standardised $N_i$, respectively.

 
 we consider the count
\begin{equation}
\nonumber
N_{k,2}:=\sum_{(x_1, \ldots , x_{k-1})\in\eta^{k-1}}
\left(\prod_{i=0}^{k-2}\bone_{\{x_i\leftrightarrow x_{i+1}\}}\right)\bone_{\{x_{k-1}\leftrightarrow x\}}
\end{equation}
of $k$-hop paths with two fixed endpoints located at $y_1=0$ and $y_2=x$. 
Note that when $k=1$, $N_{k,2}$ is a Bernoulli random variable with success rate $H(0,x)$.
On the other hand, when $k=2$,
taking $r:=k-1$ we have
$$\Pi_{\widehat{1}}[n\times 1]=\{\widehat{1}\},
\qquad n\geq 1,
$$
 hence 
\begin{equation}
\nonumber
  \kappa_n(N_{k,2})=\lambda \int_{\R^d}H(y)H(y-x)\mathrm{d}y,
 \qquad 
 n\geq 1,
\end{equation}
which implies that $N_{k,2}$ is Poisson distributed
 with parameter $\lambda \int_{\R^d}H(y)H(y-x)\mathrm{d}y$.



 \begin{table}[H] 
  \centering
% \scriptsize %   \small
  \resizebox{\textwidth}{!}{
  \begin{tabular}{|ll|ll|} % {\textwidth}{|XX|XX|}
 \hline
 \multicolumn{2}{|l}{
x,y,λ,β = var("x,y,λ,β"); assume(β$>$0)
 }
   & \multicolumn{2}{l|}{\# Variable definitions}  
 \\
 \hline
 \multicolumn{2}{|l}{
   def H(x,y,β): return exp(-β*(x-y)**2)
 } 
  & \multicolumn{2}{l|}{\# Connection function}  
 \\
 \hline
 \multicolumn{2}{|l}{
def mu(x,λ,β): return λ % *exp(-β*x**2)
}
  & \multicolumn{2}{l|}{\# Intensity}   
 \\
 \hline
 \multicolumn{2}{|l}{
G = [[1,2]]
}
  & \multicolumn{2}{l|}{\# Single vertex graph}   
 \\
 \hline
 \multicolumn{2}{|l}{
 E=[ [1] ]
} 
 & \multicolumn{2}{l|}{\# Single endpoint}   
 \\
\hline
\hline
\multicolumn{1}{|c|}{Instruction} & \multicolumn{1}{c|}{Computed quantity} & \multicolumn{1}{c|}{Output} & \multicolumn{1}{c|}{Connected non-flat partitions} 
 \\ 
 \hline
\multicolumn{1}{|c|}{c(1,1,G,E,mu,H)} & \multicolumn{1}{c|}{First cumulant} & \multicolumn{1}{c|}{$\frac{\sqrt{3} \pi {\lambda}^{2}}{3 \, {\beta}}$} & \multicolumn{1}{c|}{1} 
\\
\hline
\multicolumn{1}{|c|}{c(2,1,G,E,mu,H)} & \multicolumn{1}{c|}{Second cumulant} & \multicolumn{1}{c|}{$\frac{\sqrt{3} {\left(\sqrt{6} \pi^{{3}/{2}} {\lambda}^{3} + 2 \, \pi \sqrt{{\beta}} {\lambda}^{2}\right)}}{3 \, {\beta}^{{3}/{2}}}$} & \multicolumn{1}{c|}{33} 
\\
\hline
\multicolumn{1}{|c|}{c(3,1,G,E,mu,H)} & \multicolumn{1}{c|}{Third cumulant} & \multicolumn{1}{c|}{
$\frac{2 \, \sqrt{3} {\left(2 \, \pi^{2} {\lambda}^{4} {\left(7 \, \sqrt{15} + 30 \, \sqrt{7}\right)} + 35 \, \sqrt{3} \sqrt{\pi^{3} {\beta}} {\lambda}^{3} {\left(3 \, \sqrt{2} + 1\right)} + 70 \, \pi {\beta} {\lambda}^{2}\right)}}{105 \, {\beta}^{2}}$} & \multicolumn{1}{c|}{68} 
 \\ % [1ex]
\hline
\end{tabular}
}
    % \caption{Summary of branching tree notation.}
\end{table} 

\vspace{-0.6cm}

$$
\frac{4 \, \sqrt{2145} {\left(7 \, \sqrt{\pi} {\left(\pi^{2} {\left(\sqrt{13} {\left(\sqrt{55} + 36 \, \sqrt{3}\right)} + 24 \, \sqrt{165}\right)} {\lambda}^{5} + \sqrt{2145} \pi {\beta} {\lambda}^{3} {\left(7 \, \sqrt{2} + 6\right)}\right)} + 2 \, {\left(2 \, \sqrt{143} \pi^{2} {\left(7 \, \sqrt{3} {\left(3 \, \sqrt{2} + 4\right)} + 18 \, \sqrt{35}\right)} {\lambda}^{4} + 7 \, \sqrt{715} \pi {\beta} {\lambda}^{2}\right)} \sqrt{{\beta}}\right)}}{15015 \, {\beta}^{{5}/{2}}}
$$



\begin{table}[H] 
  \centering
% \scriptsize %   \small
  %  \resizebox{\textwidth}{!}
      {
  \begin{tabular}{|ll|ll|} % {\textwidth}{|XX|XX|}
 \hline
 \multicolumn{2}{|l}{
def mu(x,λ,β): return λ
}
  & \multicolumn{2}{l|}{\# Intensity}   
 \\
 \hline
 \multicolumn{2}{|l}{
G = [[1,2],[2,3]]
}
  & \multicolumn{2}{l|}{\# $3$-hop path} 
 \\
 \hline
 \multicolumn{2}{|l}{
 E=[[1]]
} 
 & \multicolumn{2}{l|}{\# Single endpoint}   
 \\
\hline
\hline
\multicolumn{1}{|c|}{Instruction} & \multicolumn{1}{c|}{Computed quantity} & \multicolumn{1}{c|}{Output} & \multicolumn{1}{c|}{Connected non-flat partitions} 
 \\ 
 \hline
\multicolumn{1}{|c|}{c(1,1,G,E,mu,H)} & \multicolumn{1}{c|}{First cumulant} & \multicolumn{1}{c|}{${\lambda}^{3} \sqrt{\frac{\pi^{3}}{{\beta}^{3}}}$} & \multicolumn{1}{c|}{1} 
\\ % [1ex]
\hline
\end{tabular}
}
    % \caption{Summary of branching tree notation.}
\end{table} 

\vspace{-0.6cm}

\section{Joint cumulant code}
\label{fjkldsf-2}
\noindent  


\section{Three-hop counts}
\label{appl-engineer}
\noindent
 Let $o$ denote the origin in $\R$. 
 Take $x>0$, denote $N(0,x)$ the number of $3$-hop paths connecting the origin
 to $x$ in $\Gamma(\eta_{o,x})$ on $\R$, i.e. 
\begin{equation}
  \label{3hop-1}
N(0,x):=\sum_{x_1\ne x_2\in\eta}\bone_{\{o\leftrightarrow x_1\}}\bone_{\{x_1\leftrightarrow x_2\}}\bone_{\{x_2\leftrightarrow x\}}.
\end{equation}
The moments and cumulants of $N(0,x)$ are accessible, see \cite{prkhp} for more details. 
 Denote $N_3$ the number of points containing in at least one $3$-hop connecting  to the origin $o$ in $\Gamma(\eta_0 )$, i.e.
\begin{equation}
\nonumber
 N_3:=\sum_{x\in\eta}\bone_{\{o\overset{k}{\leftrightarrow}x~\text{in}~\Gamma(\eta_o)\}}. 
\end{equation}
 By virtue of the Mecke equation \cite{LastPenrose17}, we rely on the void probability of $N(0,x)$ to obtain the expectation $\E_\lambda [ N_3 ]$, 
\begin{eqnarray}
\E_\lambda [ N_3 ] &=&\lambda\int_{\R_+}\IP\left(o\overset{k}{\leftrightarrow}x~\text{in}~\Gamma(\eta_{0,x})\right)\mathrm{d}x\nonumber\\
&=&\lambda\int_{\R_+}\IP\left(N(0,x)\geq 1\right)\mathrm{d}x\nonumber\\
&=&\lambda\int_{\R_+}\large(1-\IP\left(N(0,x)=0\right)\large)\mathrm{d}x.\label{voidprob1}
\end{eqnarray}
 Since we know about the moments and cumulants of $N(0,x)$, we may obtain the void probability via the inversion formula for characteristic functions of discrete random variables, cf. \cite[Page.~511]{feller},
$$ 
\IP(X=0) = \frac1{2\pi}\int_{-\pi}^{\pi}\varphi_X(t)\mathrm{d}t 
= \frac1{2\pi}\int_{-\pi}^\pi\exp\left(
\sum_{k=1}^\infty\frac{\gamma_k}{k!}(it)^k\right)
 \mathrm{d}t,
$$ 
where $\varphi_X$ is the characteristic function of integer-valued random variable $X$ and $\gamma_k$ stands for the $k$-th cumulant of $X$, $k\geq 1$. Alternatively, according to \cite[Corollary~1.13]{bollobas98} we have
\begin{equation}
\nonumber
  \IP(X=0)=\sum_{k=0}^{\infty}(-1)^k \frac{m_k(X)}{k!},
\end{equation}
where $m_k(X)$ is the $k$-th factorial moment of random variable $X$.

Or if $\lambda$ is large enough, we can approximate the void probability in \eqref{voidprob1} by discretised normal distribution, i.e.
\begin{equation}
\nonumber
  \IP(N(0,x)=0)\approx \Phi\left(\frac1{2\sigma}-\mu\right)-\Phi\left(-\frac1{2\sigma}-\mu\right),
\end{equation}
where $\Phi$ is the cumulative distribution of standard normal distribution and $\mu, \sigma^2$ are the mean and variance of $N(0,x)$.
 Replacing $N$ with $N(0,x)$ in \eqref{3hop-1}, we have 
\begin{eqnarray*}
  \kappa_n(N(0,x))=\sum_{\substack{\rho\in\Pi_{\widehat{1}}[n\times 2]
      \\
\rho\wedge\pi=\widehat{0}}}\int_{\R^{|\rho|}}W^{(0,x)}_\rho (\mathbf{x})\lambda^{|\rho|}\mathrm{d}\mathbf{x},
\end{eqnarray*}
where 
\begin{equation}
\nonumber
  W^{(0,x)}_\rho (x_1, \ldots ,x_{|\rho|}):=\left(\prod_{\{o,u\}\in E_2(\rho_G)}H(0,x_u)\right)\left(\prod_{\{u,v\}\in E_1(\rho_G)}H(x_u,x_v)\right)\left(\prod_{\{x,u\}\in E_2(\rho_G)}H(x_u,x)\right).
\end{equation}

\vspace{-0.3cm}

\begin{figure}[H]
\captionsetup[subfigure]{font=footnotesize}
\centering
\subcaptionbox{connected partition diagram $\Gamma(\rho,\pi)$.}[.5\textwidth]{%
\begin{tikzpicture}[scale=0.9] 
\draw[black, thick] (0,0) rectangle (5,6);
\node[anchor=east,font=\small] at (0.8,5) {1};
\node[anchor=east,font=\small] at (0.8,4) {2};
\node[anchor=east,font=\small] at (0.8,3) {3};
\node[anchor=east,font=\small] at (0.8,2) {4};
\node[anchor=east,font=\small] at (0.8,1) {5};

\node[anchor=south,font=\small] at (1,0) {o};
\node[anchor=south,font=\small] at (2,0) {1};
\node[anchor=south,font=\small] at (3,0) {2};
\node[anchor=south,font=\small] at (4,0) {x};

\filldraw [gray] (2,1) circle (2pt);
\filldraw [gray] (3,1) circle (2pt);
\filldraw [gray] (2,2) circle (2pt);
\filldraw [gray] (3,2) circle (2pt);
\filldraw [gray] (1,3) circle (2pt);
\filldraw [gray] (2,3) circle (2pt);
\filldraw [gray] (3,3) circle (2pt);
\filldraw [gray] (4,3) circle (2pt);
\filldraw [gray] (2,3) circle (2pt);
\filldraw [gray] (2,4) circle (2pt);
\filldraw [gray] (3,4) circle (2pt);
\filldraw [gray] (2,5) circle (2pt);
\filldraw [gray] (3,5) circle (2pt);
\foreach \i in {1,...,5}
         {
\draw[thick, dash dot,blue] (1,3) .. controls (1.5,\i-.5) .. (2,\i);
\draw[thick, dash dot,blue] (4,3) .. controls (3.5,\i-.5) .. (3,\i);
\draw[thick, dash dot,blue] (2,\i) .. controls (2.5,\i-.5) .. (3,\i);
} 

\draw[thick] (2,5) -- (2,4) -- (2,3);
\draw[thick] (3,3) -- (2,2) -- (2,1);

\end{tikzpicture}}%
\subcaptionbox{Diagram $\rho_G$ with $E_1$ in blue, $E_2$ in red.}[.5\textwidth]{
\begin{tikzpicture}[scale=0.9] 
\draw[black, thick] (0,0) rectangle (5,6);
\node[anchor=east,font=\small] at (0.8,5) {1};
\node[anchor=east,font=\small] at (0.8,4) {2};
\node[anchor=east,font=\small] at (0.8,3) {3};
\node[anchor=east,font=\small] at (0.8,2) {4};
\node[anchor=east,font=\small] at (0.8,1) {5};

\node[anchor=south,font=\small] at (1,0) {o};
\node[anchor=south,font=\small] at (2,0) {1};
\node[anchor=south,font=\small] at (3,0) {2};
\node[anchor=south,font=\small] at (4,0) {x};

\filldraw [gray] (2,1) circle (2pt);
\filldraw [gray] (3,1) circle (2pt);
\filldraw [gray] (2,2) circle (2pt);
\filldraw [gray] (3,2) circle (2pt);
\filldraw [gray] (1,3) circle (2pt);
\filldraw [gray] (2,3) circle (2pt);
\filldraw [gray] (3,3) circle (2pt);
\filldraw [gray] (4,3) circle (2pt);
\filldraw [gray] (2,3) circle (2pt);
\filldraw [gray] (2,4) circle (2pt);
\filldraw [gray] (3,4) circle (2pt);
\filldraw [gray] (2,5) circle (2pt);
\filldraw [gray] (3,5) circle (2pt);
\draw[thick, dash dot,red] (1,3) .. controls (1.5,3.5) .. (2,4);
\draw[thick, dash dot,red] (1,3) .. controls (1.5,2.5) .. (2,2);
\foreach \i in {1,...,5}
         {
\draw[thick, dash dot,red] (4,3) .. controls (3.5,\i-.5) .. (3,\i);
\draw[thick, dash dot,blue] (2,\i) .. controls (2.5,\i-.5) .. (3,\i);
} 

\draw[thick] (2,5) -- (2,4) -- (2,3);
\draw[thick] (3,3) -- (2,2) -- (2,1);\end{tikzpicture}}%
\caption{A example of partition diagram $\Gamma(\rho,\pi)$ and $\rho_G$ with $n=5$, $r=2$.}
\label{fig:diagram0}
\end{figure}

\vspace{1.4cm}

For any $i=1, \ldots ,r$, let 
\begin{equation}\label{neighborhood1}
  V_i :=\{j\in [r] \ : \ v_i\sim v_j~\mathrm{in}~G\}
  \quad
  \mbox{and}
  \quad
  E_i :=\left\{j\in [m] \ : \ v_i\sim e_j ~\mathrm{in}~G\right\}. 
\end{equation}

Similarly to \eqref{merge1}, we let 
\begin{equation}
\nonumber
    G_\rho(y_1, \ldots ,y_{|\rho|}):=\prod_{i=1}^nf(y_{i,1}, \ldots ,y_{i,r}),
  \qquad \rho=\{\rho_1, \ldots ,\rho_h\}\in \Pi ([n]\times[r]), 
\end{equation}
and define the function $F$ from $\bigcup_{n=1}^{\infty}\Pi ([n]\times[r])$ to $\R$
as 
\begin{equation}
  \nonumber
  F(\rho):=\lambda^{|\rho|}\int_{(\R^d)^{|\rho|}}\E_\lambda \left[ G_\rho(\mathbf{y})\right]\mathrm{d}\mathbf{y},
  \qquad
 \rho\in \Pi ([n]\times[r]).
\end{equation}
\begin{prop}
 The cumulants of $N^G_{y_1}$ admit the following expression:  
 \begin{equation}
\nonumber % \label{connectedcumulant}
   \kappa_n(N_G)=\sum_{\substack{\sigma\in\Pi_{\widehat{1}} ([n]\times[r])
       \\\sigma\wedge\pi=\widehat{0}}}F(\sigma),
  \qquad n\geq 1.
\end{equation}
\end{prop}


, and denote by
\begin{equation}
\nonumber
  D_i:=|B_i^{[r]}|+|B_i^{[m]}|,
\end{equation}
the degree of vertex $v_i$,
$i=1,\ldots , r+m$. Let also 
\begin{equation}\label{maxdegree1}
  M^{[r]}:=\max_{i=1,\ldots , r}
  |B_i^{[r]}|,
  \qquad
  M^{[m]}:=\max_{i=1,\ldots , r} |B_i^{[m]}|,
\end{equation}
denote the maximal in-degree and out-degree of the vertices $v_1, \ldots ,v_r$, and $\widebar{m}:=\min_{i=1,\ldots , r} |B_i^{[m]}|$, their smallest out-degree. 


\def\cprime{$'$} \def\polhk#1{\setbox0=\hbox{#1}{\ooalign{\hidewidth
  \lower1.5ex\hbox{`}\hidewidth\crcr\unhbox0}}}
  \def\polhk#1{\setbox0=\hbox{#1}{\ooalign{\hidewidth
  \lower1.5ex\hbox{`}\hidewidth\crcr\unhbox0}}} \def\cprime{$'$}
\begin{thebibliography}{BRSW17}

%\bibitem[BKR89]{BKR}
%A.D. Barbour, M.~Karo{\'n}ski, and A.~Ruci{\'n}ski.
%\newblock A central limit theorem for decomposable random variables with
%  applications to random graphs.
%\newblock {\em J. Combin. Theory Ser. B}, 47(2):125--145, 1989.
%
%\bibitem[BRSW17]{bogdan}
%K.~Bogdan, J.~Rosi\'{n}ski, G.~Serafin, and L.~Wojciechowski.
%\newblock L\'{e}vy systems and moment formulas for mixed {P}oisson integrals.
%\newblock In {\em Stochastic analysis and related topics}, volume~72 of {\em
%  Progr. Probab.}, pages 139--164. Birkh{\"{a}}user/Springer, Cham, 2017.
%
\bibitem[Bol98]{bollobas98} B. Bollob\'as. Random graphs. In Modern graph theory (pp. 215-252). Springer, New York, NY, 1998.
%\bibitem[CT22]{can2022}
%V.~H. Can and K.~D. Trinh.
%\newblock Random connection models in the thermodynamic regime: central limit
%  theorems for add-one cost stabilizing functionals.
%\newblock {\em Electron. J. Probab.}, 27:1--40, 2022.
%
\bibitem[DE13]{doring}
H.~D{\"{o}}ring and P.~Eichelsbacher.
\newblock Moderate deviations via cumulants.
\newblock {\em J. Theoret. Probab.}, 26:360--385, 2013.

\bibitem[DJS22]{doering}
H.~D{\"o}ring, S.~Jansen, and K.~Schubert.
\newblock The method of cumulants for the normal approximation.
\newblock {\em Probab. Surv.}, 19:185--270, 2022.

%\bibitem[ER59]{ER}
%P.~Erd{\Horig{o}}s and A.~R\'enyi.
%\newblock On random graphs. {I}.
%\newblock {\em Publ. Math. Debrecen}, 6:290--297, 1959.
%
%\bibitem[ET14]{eichelsbacher}
%P.~Eichelsbacher and C.~Th{\"a}le.
%\newblock New {B}erry-{E}sseen bounds for non-linear functionals of {P}oisson
%  random measures.
%\newblock {\em Electron. J. Probab.}, 19:no. 102, 25, 2014.
%
\bibitem[Feller71]{feller}
W.~Feller.
\newblock An introduction to probability theory and its applications, vol 2. 
\newblock {\em John Wiley \& Sons}, 1971.
%\bibitem[Gil59]{G}
%E.N. Gilbert.
%\newblock Random graphs.
%\newblock {\em Ann. Math. Statist}, 30(4):1141--1144, 1959.
%
%\bibitem[GT18a]{grotethale18}
%J.~Grote and C.~Th{\"a}le.
%\newblock Concentration and moderate deviations for {P}oisson polytopes and
%  polyhedra.
%\newblock {\em Bernoulli}, 24:2811--2841, 2018.
%
%\bibitem[GT18b]{thale18}
%J.~Grote and C.~Th{\"a}le.
%\newblock Gaussian polytopes: a cumulant-based approach.
%\newblock {\em J. Complexity}, 47:1--41, 2018.
%
%\bibitem[Kho08]{khorunzhiy}
%O.~Khorunzhiy.
%\newblock On connected diagrams and cumulants of {E}rd{\Horig{o}}s-{R}\'enyi
%  matrix models.
%\newblock {\em Comm. Math. Phys.}, 282:209--238, 2008.
%
%\bibitem[KRT17]{reichenbachsAoP}
%K.~Krokowski, A.~Reichenbachs, and C.~Th{\"a}le.
%\newblock Discrete {M}alliavin-{S}tein method: {B}erry-{E}sseen bounds for
%  random graphs and percolation.
%\newblock {\em Ann. Probab.}, 45(2):1071--1109, 2017.
%
%\bibitem[LNS21]{LNS21}
%G.~Last, F.~Nestmann, and M.~Schulte.
%\newblock The random connection model and functions of edge-marked {P}oisson
%  processes: second order properties and normal approximation.
%\newblock {\em Ann. Appl. Probab.}, 31(1):128--168, 2021.
%
\bibitem[LP17]{LastPenrose17}
G.~Last and M.D. Penrose.
\newblock {\em Lectures on the {P}oisson process}, volume~7 of {\em Institute
  of Mathematical Statistics Textbooks}.
\newblock Cambridge University Press, Cambridge, 2017.
%
%\bibitem[LRR16]{lachieze-rey}
%R.~Lachi\`eze-Rey and M.~Reitzner.
%\newblock {$U$}-statistics in stochastic geometry.
%\newblock In G.~Peccati and M.~Reitzner, editors, {\em Stochastic Analysis for
%  {P}oisson Point Processes: {M}alliavin Calculus, {W}iener-{I}t{\^o} Chaos
%  Expansions and Stochastic Geometry}, volume~7 of {\em Bocconi \& Springer
%  Series}, pages 229--253. Springer, Berlin, 2016.
%
\bibitem[LP22]{LiuPrivault}
Q.~Liu and N.~Privault.
\newblock Normal approximation of subgraph counts in the random-connection model.
\newblock Preprint arXiv:2301.12145, 26 pages, 2023.
%\bibitem[MM91]{MalyshevMinlos91}
%V.A. Malyshev and R.A. Minlos.
%\newblock {\em Gibbs random fields}, volume~44 of {\em Mathematics and its
%  Applications (Soviet Series)}.
%\newblock Kluwer Academic Publishers Group, Dordrecht, 1991.
%
%\bibitem[Pri12]{momentpoi}
%N.~Privault.
%\newblock Moments of {P}oisson stochastic integrals with random integrands.
%\newblock {\em Probab. Math. Statist.}, 32(2):227--239, 2012.
%
\bibitem[Pri19]{prkhp}
N.~Privault.
\newblock Moments of $k$-hop counts in the random-connection model.
\newblock {\em J. Appl. Probab.}, 56(4):1106--1121, 2019.
%
\bibitem[Pri22]{privaultkhops}
N.~Privault.
\newblock Asymptotic analysis of $k$-hop connectivity in the 1{D} unit disk
  random graph model.
\newblock Preprint arXiv:2203.14535, 40 pages, 2022.
%
%\bibitem[PS20]{PS2}
%N.~Privault and G.~Serafin.
%\newblock Normal approximation for sums of discrete {$U$}-statistics -
%  application to {K}olmogorov bounds in random subgraph counting.
%\newblock {\em Bernoulli}, 26(1):587--615, 2020.
%
%\bibitem[PS22]{PS4}
%N.~Privault and G.~Serafin.
%\newblock Berry-{E}sseen bounds for functionals of independent random
%  variables.
%\newblock {\em Electron. J. Probab.}, 27:1--37, 2022.
%
%\bibitem[PT11]{peccatitaqqu}
%G.~Peccati and M.~Taqqu.
%\newblock {\em Wiener Chaos: Moments, Cumulants and Diagrams: A survey with
%  Computer Implementation}.
%\newblock Bocconi \& Springer Series. Springer, 2011.
%
%\bibitem[R{\"o}l22]{roellin2}
%A.~R{\"o}llin.
%\newblock Kolmogorov bounds for the normal approximation of the number of
%  triangles in the {E}rd{\Horig{o}}s-{R}\'enyi random graph.
%\newblock {\em Probab. Engrg. Inform. Sci.},
%  36(3):747--773, 2022.
%
%\bibitem[RSS78]{rudzkis}
%R.~Rudzkis, L.~Saulis, and V.A. Statuljavi\v{c}us.
%\newblock A general lemma on probabilities of large deviations.
%\newblock {\em Litovsk. Mat. Sb.}, 18(2):99--116, 217, 1978.
%
%\bibitem[Ruc88]{rucinski}
%A.~Ruci{\'n}ski.
%\newblock When are small subgraphs of a random graph normally distributed?
%\newblock {\em Probab. Theory Related Fields}, 78:1--10, 1988.
%
\bibitem[SS91]{saulis}
L.~Saulis and V.A. Statulevi\v{c}ius.
\newblock {\em Limit theorems for large deviations}, volume~73 of {\em
  Mathematics and its Applications (Soviet Series)}.
\newblock Kluwer Academic Publishers Group, Dordrecht, 1991.
\newblock Translated and revised from the 1989 Russian original.

\end{thebibliography}
